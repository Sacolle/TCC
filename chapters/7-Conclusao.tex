% 7. Conclusão
\chapter{Conclusão}
\label{chap7}

Este projeto se iniciou visando definir um sistema para provar métodos de segurança de memória. Esse desenvolvimento estendeu-se até a definição da linguagem de \emph{front-end}, $PCL_{front}$. Com ela, pode-se definir uma semântica que evita falhas de \emph{Use-After-Free} utilizando um sistema de \emph{Borrow Checker} com o algoritmo Polonius. Esse sistema se inspirou também em outros métodos para adequá-lo melhor ao contexto de implementação.

A construção de $PCL_{back}$ se mostrou expressiva o suficiente para modelar todas as falhas de memória consideradas com clareza. Durante este texto, foram usados alguns exemplos dessa expressividade (Programa \ref{lst:pclback:spacial}), levando em conta a ocorrência dessas falhas em C. Mais exemplos foram construídos reproduzindo falhas de C em $PCL_{back}$, mas foram omitidos por espaço. Uma série deles existe na publicação Erros de Memória e Porcelain no blog do autor\footnote{\url{https://sacolle.github.io/blog/posts/porcelain-emulando-os-erros-de-memoria-de-c/}}, que foi desenvolvido paralelamente a este trabalho.

Este projeto serve como um estudo sobre falhas de memória, suas definições, problemas e soluções. Ao construir as linguagens $PCL_{back}$ e $PCL_{front}$ pode-se aproximar mais desses conceitos, aplicando-os de forma prática e direta. Nesse processo, almejou-se definir mais precisamente o que é uma falha de memória e, dessa forma, construir novas técnicas para a mitigação destas. A relevância deste projeto é essa visão mais micro dessas falhas, as suas minúcias e desafios quando as tenta solucionar.

% falar aqui da do sistema em haskell
\section{Resultados Obtidos}

A AST das linguagens, e as suas respectivas avaliações, foram implementadas em Haskell. A sintaxe da linguagem se provou muito próxima da notação matemática, permitindo rápida iteração entre mudança de notação e mudança de código e vice-versa. Para $PCL_{back}$ foi também desenvolvido um \emph{Lexer} e um \emph{Parser}, também em Haskell, que auxiliaram muito na prototipação da linguagem e na formulação de exemplos. A adição de declarações globais surgiu desse processo de iteração, em que se observou que seria necessário para demonstrar as falhas associadas a estruturas globais de alocação de memória. A formatação das linguagens também as posiciona bem caso o projeto progrida para o uso de um sistema de provas formais, como Coq \cite{COQ} ou Lean \cite{LEAN4}, que usam uma notação funcional similar a Haskell.
% 	7.1 Limitações
\section{Limitações}

Devido às limitações de escopo deste projeto, não foi possível atingir todas as suas metas. Mesmo com $PCL_{front}$ tendo sua funcionalidade reduzida para caber no escopo, não foi possível chegar no estágio de validação formal dessa linguagem. Os testes realizados servem como uma boa base para os locais que devem ser explorados, mas precisa-se, ainda, aumentar a cobertura destes e validar o modelo de \emph{alias} da linguagem formalmente. Por consequência, não se conseguiu progredir para a etapa de prova do trabalho, em que se tentaria provar formalmente a segurança de memória da linguagem $PCL_{front}$ quanto erros de \emph{Use-After-Free} através da compilação para $PCL_{back}$.

A linguagem $PCL_{back}$, por sua vez, modelou bem os falhas da linguagem C, mas faltou em alguns aspectos. A falta de um sistema de tipos simplificou a implementação, mas impediu a capacidade de detecção de certos erros menos comuns. Um deles é a indexação incorreta de estruturas dentro de \emph{buffers}, ou de listas aninhadas. Nesse caso, pode-se não intencionalmente sobrescrever membros de uma estrutura da lista ao acessar outro, gerando um \emph{Out-Of-Bounds-Acess} para a subestrutura dentro do \emph{buffer}. Além disso, o fato do sistema de alocações $\KW{free}$ e $\KW{malloc}$ não usarem alocações globais pode ser um problema para a reprodução de alguns exemplos, mas nenhum que se tenha realizado durante o projeto.

% 	7.2 Future Work
\section{Trabalhos Futuros}

O futuro deste trabalho está bem traçado, complementar e validar $PCL_{front}$, possivelmente reformulando alguns conceitos, como a ampla mutabilidade dos \emph{alias}, e depois provar que o sistema previne falhas de \emph{Use-After-Free}. É de interesse também demonstrar que a compilação para $PCL_{back}$ é correta. Não é fora do escopo talvez gerar uma linguagem de \emph{front-end} que implemente uma solução de memória mais simples para testar o processo e fazer modificações necessárias nele, caso a prova do sistema de $PCL_{front}$ se demonstre muito penosa.

Uma aspiração grandiosa deste trabalho é fazer várias linguagens de \emph{front-end}, refinando o trabalho todo no processo de prová-las através da compilação para $PCL_{back}$. Qualquer progresso no trabalho será publicado no blog do Autor\footnote{\url{https://sacolle.github.io/blog/}}, que com sorte será periodicamente. 