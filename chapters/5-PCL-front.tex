% 5. PCL front
\chapter{PCL-front}

% The front-end language, $PCL_{front}$, aims to model a borrow checking system. 
% However, due to the shape of $PCL_{back}$ and to ease compilation, 
% two design decisions were taken that change the shape of $PCL_{front}$'s borrow checker.
% $\mathbf{1\!\!:}$ Instead of using affine types, 
% with automatic \textit{drop} call at the end of the scope, 
% $PCL_{front}$ uses linear types\cite{CSLINLOG, Wadler1990LinearTC}, 
% with the pointer type $\text{*}\tau$ being linear whilst the others are nonlinear. 
% This decision simplifies compilation and also models Rust's \textit{Copy} and 
% \textit{Clone} subtextual fling with linearity. $\mathbf{2\!\!:}$ Without parallelism, 
% it lessens the necessity of only one mutable reference or, exclusively, many immutable 
% references to a given pointer. With that, $PCL_{front}$ allows for multiple mutable 
% references, called \textit{alias} ($@\tau'a$) in the type system.

% The aim with this language is, when compiled to $PCL_{back}$, to never reach 
% a derivation of $\KW{panic}\;\KW{UseAfterFree}$. The syntax of the language follows:

% 	5.1 PCL front's Borrow Checker
\section{O Borrow Checker de PCL-front}

% 	5.2 Syntax
\section{Sintaxe}

\begingroup
\setlength{\jot}{-0.2ex} 
	\begin{align*}
		Types \ni \tau ::&= \KW{int} \OR \text{*}\tau \OR @\tau'a \OR (\bar\tau) \to \tau \\
		Locals \ni l ::&= l^m \OR l^p &&\\ 
		Value \ni v ::&= n \OR l && \\
		BinOp \ni op ::&= + | - | * | < | > | = | \land | \lor \\
		Expression \ni e ::&= x \OR v \\
		&\NLOR e\; op\; e \OR !e  \\
		&\NLOR \text{*}e \OR \&x \OR \KW{alias} \; x \OR \KW{alias}\text{*} \; x\\
		&\NLOR e;e \OR \{e\} \\ 
		&\NLOR \KW{new}\typearg\tau(e) \OR \KW{delete}(e, e) \; \\ 
		&\NLOR \KW{var}\; x: \tau := e \OR e_1 := e_2 \\
		&\NLOR e_1 :=: e_2 \OR e_1 :=:\!\text{*}\;e_2\\
		&\NLOR \KW{if}(e) \; e \; \KW{else} \; e \; \OR \KW{while}(e) \; e \\
		&\NLOR f(\bar e) \\ 
		&\NLOR \KW{stop}\\ 
		&\NLOR \KW{nullprt}\typearg{\tau} \OR \KW{nullalias}\typearg{\tau}\\ 
		Function \ni F ::&= \KW{fn} \; f(\overline{x : \tau}) \to \tau \; e \; F \; | \; \mathbf{let}\;() \; e \\
	\end{align*}
\endgroup

% $'a$, $x$ and $f$ are meta variables ranging over regions $r \in Regs$ in the program, 
% variable names and function names respectively. $x$ is also used as a loan name 
% $x \in Loans$, given that loans are from variable names.
 
% One important aspect of $PCL_{front}$'s linear type system is how to deal with pointer
% dereferences. Following Cyclone's rules for accessing linear pointers\cite{CYCLONEMEM},
% a pointer $\text{*}\tau$ or $@\tau'a$ can be safely dereferenced if $\tau$ is nonlinear.
% For other cases a swap and alias attached with a dereference are safe abstractions 
% ($\mathbf{alias}\text{*}\;e$, $e_1 :=:\!\!\text{*} \;e_2$).

% There are a few omissions of note. There are no tuple or recursive types. 
% Even thought $PCL_{back}$ could represent them, they complicate the borrow checker 
% and the compilation step. However the omission is mostly due to time constraints. 
% This is the same reason for the missing of $\mathbf{while}$ and global declarations. 


% 	5.3 Checagem de Tipos
\section{Checagem de Tipos e Regiões}

% The borrow checker implemented is based on the Polonius 
% proposal\cite{Matsakis_2018, Stjerna1684081}. This implementation requires that 
% a node in the derivation tree alter a state which will be used for the sequent 
% assessment of other nodes. This imposes an evaluation order of depth first, 
% which is shown in the notation by each assessment returning its computed type 
% and its resulting environments. 

% These environments are: $\Gamma$, the association between names and types; 
% $U$, similar to $\Gamma$, but as an intermediary space between used value and 
% restored value; $R$ is a set of region pairs $('a, 'b)$, such as 
% $'a : 'b$ ($a$ outlives $b$), $R$ is transitive; $L$ is a set of region to 
% loan relations, which propagates over $R$ in that 
% $('a,'b) \in R \land ('a, x) \in L\to ('b, x) \in L$. Any access to an alias
% $@\tau'a$ needs to validate that $'a$ has a valid loan, as in (\textsc{DerefAlias}). 


%   5.4 Definição via compilação
\section{Definição via Compilação}