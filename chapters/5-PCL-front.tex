% 5. PCL front
\chapter{PCL-front}
\label{chap5}

O propósito deste projeto é utilizar a linguagem $PCL_{back}$ como um meio de prova para estratégias de solução de memória. Para demonstrar essa capacidade, desenvolveu-se a linguagem de \emph{front-end}, $PCL_{front}$, que implementa um sistema de \emph{Borrow Checker} como elaborado no Capítulo \ref{chap4}. A linguagem $PCL_{front}$ é definida por meio de sintaxe formal, sistema de tipos (o qual incorpora as regras de validação do \emph{Borrow Checker}) e semântica de compilação para $PCL_{back}$.

O \emph{design} de $PCL_{front}$ partiu do formato de $PCL_{back}$, focando em gerar o menor número de modificações sintáticas para que a compilação fosse simples de se definir. Duas decisões foram tomadas que alteraram o funcionamento do \emph{Borrow Checker} de $PCL_{back}$ comparado com o de Rust, a fim de manter a simplicidade da linguagem. (a) Ao invés de se utilizar um sistema de Rust que evoca tipos lineares, decidiu-se usá-los propriamente, voltando a alocações e desalocações manuais, regidas pela regra do uso exatamente uma vez. Na linguagem, o tipo $\text{*}\tau$ é linear enquanto os demais são não lineares. Isso simplifica a compilação por evitar as chamadas implícitas de liberação de memória que existem em Rust, mas ainda modela o comportamento deste, com liberações sendo necessárias no fim de escopos para que um programa seja bem tipado. (b) Sem paralelismo, há menos necessidade de utilizar a regra de apenas um empréstimo mutável ou, exclusivamente, vários empréstimos imutáveis. Assim, $PCL_{front}$ permite várias referências mutáveis, chamadas \emph{alias} ($@\tau'a$) no sistema de tipos. 

O objetivo dessa linguagem é, quando um dado programa $PCL_{front}$ é compilado para um código $PCL_{back}$, esse código nunca atinge uma derivação de $\KW{panic}$ $\KW{UseAfterFree}$. As demais falhas temporais são permitidas na atual configuração do programa. Ao se aproximar mais do sistema de linearidade de Rust, movendo-se para um sistema de liberação automática de memória, pode-se começar a validar as demais falhas de memória temporais.

% 	5.1 Syntax
\section{Sintaxe}

A sintaxe da linguagem é bem próxima de um $PCL_{back}$ tipado. O modelo de linearidade necessitou de alguns operadores adicionais, sendo eles a troca $:=:$ e troca com dereferência $:=:\!\!\text{*}$; o incremento $+\!\!=$ e decremento $-\!\!=$; e a cópia de ponteiro $\KW{alias}$ e a cópia de ponteiro que dereferência $\KW{alias}\!\text{*}$. Os termos $\KW{new}$ e $\KW{delete}$ substituem o $\KW{malloc}$ e $\KW{free}$ com a mesma funcionalidade, mas com nomes que indicam uma operação de mais alto nível. Por fim, os \emph{macros} \emph{PANIC} e \emph{NULL} se tornaram as expressões $\KW{stop}$ e $\KW{nullptr}\typearg{\tau}$ respectivamente. Tem-se também $\KW{nullalias}\typearg{\tau}$ como o \emph{alias} nulo da linguagem. 

\def\notmov#1{\color{blue}\underline{{\color{black}#1}}\color{black}}

\begin{figure}[ht]
	\caption{Sintaxe de $PCL_{front}$}
	\label{fig:pcl-front:sintax}
	\begin{align*}
		Types \ni \tau ::&= \KW{int} \OR \text{*}\tau \OR @\tau'a \OR (\bar\tau) \to \tau \\
		Locals \ni l ::&= l^m \OR l^p &&\\ 
		Value \ni v ::&= n \OR l && \\
		BinOp \ni op ::&= + | - | * | < | > | = | \land | \lor \\
		Expression \ni e ::&= x \OR v \\
		&\NLOR e\; op\; e \OR !e \OR \notmov{x +\!\!= e} \OR \notmov{x -\!\!= e_2}\\
		&\NLOR e;e \OR \{e\} \\ 
		&\NLOR \KW{let}\;x:\text{*}\!\tau = \KW{new}(e) \OR \KW{delete}(x, e) \; \\ 
		&\NLOR \KW{var}\; x: \tau := e \OR e_1 := e_2 \\
		&\NLOR \notmov{\text{*}e} \OR \notmov{\&x} \OR \notmov{\KW{alias} \; x} \OR \notmov{\KW{alias}\text{*} \; x}\\
		&\NLOR \notmov{x_1 :=: x_2} \OR \notmov{x_1 :=:\!\text{*}\;x_2}\\
		&\NLOR \KW{if}(e) \; e \; \KW{else} \; e\\
		&\NLOR \KW{let}\;x:\tau = f(\bar e) \\ 
		&\NLOR \KW{stop}\\ 
		&\NLOR \KW{nullprt}\typearg{\tau} \OR \KW{nullalias}\typearg{\tau}\\ 
		Function \ni F ::&= \KW{fn} \; f(\overline{x : \tau}) \to \tau \; e \; F \; | \; \KW{fn}\;() \; e \\
	\end{align*}
	\vspace{-4em}
	\begin{flushleft}
		\small
		{\color{red}*}Elementos subscritos em {\color{blue} azul} não movem valores.
	\end{flushleft}
	\legend{Fonte: Os Autores.}
\end{figure}

A sintaxe é descrita na Figura \ref{fig:pcl-front:sintax}. As meta-variáveis \noindent$'a$, $x$ e $f$ cobrem as regiões, nomes de variáveis e nomes de funções de um programa, respectivamente. Os nomes de variáveis $x$ também são usados como o nome dos empréstimos, visto que nesta versão do \emph{Borrow Checker} não se precisa anotar o modo de acesso: todos os acessos são mutáveis.
 
Há algumas omissões relevantes. Não há tipos tupla ou recursivos. Mesmo que $PCL_{back}$ pudesse representar eles, eles complicam o \emph{Borrow Checker} e o passo de compilação. Entretanto, a omissão é principalmente por limite de tempo e escopo do projeto. Essa é a mesma razão pela omissão do operador $\KW{while}$ e declarações globais.


\section{Linearidade e \emph{Alias}}

O sistema de \emph{Borrow Checker} de $PCL_{front}$ é fortemente conectado com o seu sistema de linearidade. O tipo ponteiro, $\text{*}\tau$, é linear e aponta para uma região da \emph{heap}. Ele é obtido através da função $\KW{new}$ e só é apagado através da função $\KW{delete}$. Os demais tipos são não lineares. O tipo \emph{alias}, $@\tau'a$ é uma forma de copiar o tipo ponteiro, e também é usado para referenciar valores na pilha. Ele denota uma cópia (\emph{alias}) de um ponteiro e é equivalente aos empréstimos de Rust. Uma consequência da modelagem com tipos lineares é o registro de quais operações consomem o valor. Isto é, quais operações que, passando uma variável $x$ do tipo $\text{*}\!\tau$, consomem o valor e o tornam inacessível via $x$. As operações subscritas em azul não movem os valores passados a eles. Excluindo os operadores de dereferência (*) e referência (\&), são todos operadores novos, introduzidos com o propósito de não mover os valores no uso. Por exemplo, os operadores de incremento e decremento foram criados para poder-se aumentar o valor de um ponteiro sem que se use o valor, dessa forma acessos a listas em memória ficam mais fáceis, como em \ref{fig:pcl-front:inc}.

\begin{figure}[ht]
	\caption{Acesso a listas em memória em $PCL_{front}$}
	\label{fig:pcl-front:inc}
	\begin{lstlisting}[language=PCLfront]
do(*x);
x += 1;
do(*x);
x += 1;
// ...
	\end{lstlisting}
	\legend{Fonte: Os Autores.}
\end{figure}

Porém, há um problema nesse sistema, especificamente com o fato que o ato de dereferenciar um ponteiro não o consome. No exemplo \ref{fig:pcl-front:uaf}, consegue-se duplicar um ponteiro a memória e liberá-lo, realizando uma falha de \emph{Use-After-Free} dentro da linguagem feita para preveni-los.

\begin{figure}[ht]
	\caption{Geração de erro de \emph{Use-After-Free} com $PCL_{front}$ sem restrições.}
	\label{fig:pcl-front:uaf}
	\begin{lstlisting}[language=PCLfront]
let a: **int = new<*int>(1);
let b: *int = new<int>(1);
*a := b;
//acessa o ponteiro salvo no |.{\color{codegreen}endereço}. a
delete(*a, 1);
//acessa novamente
*a // <- use after free
	\end{lstlisting}
	\legend{Fonte: Os Autores.}
\end{figure}

A operação de dereferência de valores precisa então ser restringida. Dessa forma, tem-se que uma operação de deferência $\text{*}\!e$ só é válida se $e$ for de um tipo não linear, $e\!:\,!\tau$ ($!$ prefixando um tipo indica que este é não linear). Essa limitação justifica os operadores $:=:\!\!\text{*}$ e $\KW{alias}\text{*}$, operações em que mesmo com um tipo $e$ linear, pode-se dereferenciar, pois a operação imediata após é segura e absorve o valor. Essa dinâmica vem inspirada nos \emph{tracked pointers} de Cyclone \cite[p.6]{CYCLONEMEM}, que limitam também o acesso a ponteiros alinhados da mesma forma. Um último adicional é que a função $\KW{new}$ inicializa a memória com o valor base do tipo indicado. Dessa forma \ref{fig:pcl-front:uaf} também está incorreto na sua terceira linha, pois descarta o tipo linear em memória. A limitação sintática de troca ser só entre variáveis existe por esse motivo. O trecho \ref{fig:pcl-front:init} descreve essa inicialização mais propriamente. O $\KW{delete}$ na linguagem pode consumir ponteiros nulos sem emissão de falhas.

\begin{figure}[ht]
	\caption{Inicialização correta de ponteiros aninhados com $PCL_{front}$.}
	\label{fig:pcl-front:init}
	\begin{lstlisting}[language=PCLfront]
let a: **int = new<*int>(1); //|.{\color{codegreen}Inicia espaço alocado com nullptr}.
let b: *int = new<int>(1); //|.{\color{codegreen}Inicia espaço alocado com 0}.
*b := 2;
//|.{\color{codegreen}Troca o ponteiro $b$ pelo ponteiro na posição $a$}. 
b :=:* a;
delete(b, 1);
//|.{\color{codegreen}$a$ agora é $a \to ptr \to 2$}. 
	\end{lstlisting}
	\legend{Fonte: Os Autores.}
\end{figure}

O \emph{Borrow Checker} da linguagem, então, é extremamente similar ao modelo do Polonius, descrito em \ref{sec:chap4:polonius}, apenas com nomes diferentes. Qualquer operação que use, consumindo ou não, o ponteiro, invalida os seus \emph{alias}.
% 	5.3 Checagem de Tipos
\section{Checagem de Tipos e Regiões}

A checagem de tipos dessa linguagem inclui a validação de acessos a \emph{alias}, ou seja, o \emph{Borrow Checker} faz parte do sistema de tipos. Uma consequência disso é que acessos a um nó da árvore de avaliação podem ter efeitos colaterais em nós irmãos e primos, necessitando de uma ordem de avaliação em profundidade para que todos os nós sejam computados com a informação de todos os nós anteriores. Dessa forma, a avaliação de um termo devolve o seu tipo e o ambiente resultante na computação, na forma $\Delta \vdash e : \tau\;|\;\Delta'$, em que $e$ computado sobre $\Delta$ tem o tipo $\tau$ e resulta no ambiente $\Delta'$.

Os ambientes usados para computação são $\Gamma$, $U$, $R$ e $L$. O ambiente $\Gamma$ é o ambiente de variáveis para tipos, com $U$ sendo o seu coadjuvante. Ele também é um ambiente de variáveis para tipos, mas é o local em que variáveis acessadas são movidas para que a operação imediatamente anterior a elas decida se serão descartadas ou reincorporadas em $\Gamma$. Os ambientes $R$ e $L$ são dedicados a avaliação dos \emph{alias} do programa, contendo o mesmo significado que o descrito em \ref{sec:chap4:polonius}, $R$ é a relação entre regiões e $L$ é a relação de regiões para conjuntos de concessões. 

As regras de tipo usam algumas funções auxiliares que precisam ser definidas. Uma delas é a $\FN{newlabel}$, que retorna uma região única nova ao programa. A função $\FN{live}('r, L)$ (\ref{fig:def:live}) retorna verdadeiro se a região $'r$ está viva naquele ponto do programa, verificando se existe alguma concessão no conjunto de $'r$ em $L$. O operador auxiliar complementar ao $\FN{live}$ é o $\FN{rmLoan}(x, L)$ (\ref{fig:def:rmloan}), que gera um novo $L$ sem as regiões que possuem uma concessão a $x$.

As funções de operação binária usam a função auxiliar $\FN{addSubType}(\tau_1, \tau_2)$ (\ref{fig:def:addsub}) para simplificar a notação. Ela retorna o tipo que corresponde uma operação entre os seus dois operandos. Há correspondência se $\tau_1$ e $\tau_2$ forem $\KW{int}$, resultando em $\KW{int}$ ou se um deles for $\KW{int}$ e o outro sendo tipo ponteiro ($\text{*}\!\tau$) ou tipo \emph{alias} ($@\tau'r$), resultando nos tipos ponteiro e \emph{alias} respectivamente. 

\begin{figure}[ht]
	\caption{Função que determina se uma região $'r$ está viva em $L$.}
	\label{fig:def:live}
	\begin{align}
		&\FN{live}('r, L) = \exists ('w, l^s) \in L, {'r} = {'w} \land l^s \neq \emptyset
	\end{align}
	\legend{Fonte: Os Autores.}
\end{figure}

\begin{figure}[ht]
	\caption{Função que gera um $L'$ sem todas as regiões $'r$ de $L$ com uma concessão a $x$.}
	\label{fig:def:rmloan}
	\begin{align}
		&\FN{rmLoan}(x, L) = \{(r, l^s) \OR \forall (r, l^s) \in L, x \notin l^s \}
	\end{align}
	\legend{Fonte: Os Autores.}
\end{figure}

\begin{figure}[ht]
	\caption{Função que gera o tipo resultante entre $\tau_1$ e $\tau_2$ no caso de soma ou subtração.}
	\label{fig:def:addsub}
	\begin{align}
		&\FN{addSubType}(\KW{int}, \KW{int}) = \KW{int}\\ 
		&\FN{addSubType}(\KW{int}, \text{*}\!\tau) = \text{*}\!\tau\\ 
		&\FN{addSubType}(\text{*}\!\tau, \KW{int}) = \text{*}\!\tau\\ 
		&\FN{addSubType}(\KW{int}, @\tau'r) = @\tau'r\\ 
		&\FN{addSubType}(@\tau'r, \KW{int}) = @\tau'r
	\end{align}
	\legend{Fonte: Os Autores}
\end{figure}

Uma característica da ausência do tipo tupla é que não há a existência do tipo unitário por consequência. Decidiu-se não o adicionar por simplicidade. Dessa forma, algumas regras como \hyperref[trule:var]{(VAR)}, \hyperref[trule:delete]{(DELETE)} e \hyperref[trule:swap]{(SWAP)} retornam $\KW{int}$ como forma de retorno do tipo unitário. Na parte de compilação \ref{chap5:compile}, essas operações retornarão o número 1. Dessa forma, a seguir se tem as regras de tipo de $PCL_{front}$:

\labelRule{\label{trule:number}}{
	\infrule[Number]
		{}
		{\Gamma;U;R;L \vdash n : \KW{int} \OR \Gamma;U;R;L} 
}

\labelRule{\label{trule:nonlin-var}}{
	\infrule[Var-nonlin]
		{}
		{\Gamma, x:\,!\tau;U;R;L \vdash x : \tau \OR \Gamma,x:\tau;U;R;L} 
}
\labelRule{\label{trule:lin-var}}{
	\infrule[Var-lin]
		{}
		{\Gamma, x:\tau;U;R;L \vdash x : \tau \OR \Gamma;U, x:\tau;R;\FN{rmLoan}(x, L)}
}

\labelRule{\label{trule:ref}}{
	\infrule[Ref]
		{'r = \textbf{newlabel}()}
		{\Gamma, x:\tau;U;R;L \vdash \&x : @\tau'r \OR \Gamma,x:\tau;U;R;L, (r, \{ x \})}
}

\labelRule{\label{trule:binop-addsub}}{
	\infrule[Binop-AddSub]
		{\Gamma;U;R;L \vdash e_1: \tau_1 \OR \Gamma';U';R';L' \\ 
		\Gamma';\emptyset;R';L' \vdash e_2 : \tau_2 \OR \Gamma'';U'';R'';L''\\
		op \in \{+, -\} \quad \tau_r = \FN{addSubType}(\tau_1, \tau_2)
		}
		{\Gamma;U;R;L \vdash e_1\;op\;e_2: \tau_r \OR \Gamma'';\emptyset;R'';L''} 
}

\labelRule{\label{trule:binop-comp}}{
	\infrule[Binop-Comp]
		{\Gamma;U;R;L \vdash e_1: \tau \OR \Gamma';U';R';L' \\ 
		\Gamma';\emptyset;R';L' \vdash e_2 : \tau \OR \Gamma'';U'';R'';L''\\
		op \in \{=, <, >\}
		}
		{\Gamma;U;R;L \vdash e_1\;op\;e_2: \tau \OR \Gamma'';\emptyset;R'';L''} 
}

\labelRule{\label{trule:binop}}{
	\infrule[Binop]
		{\Gamma;U;R;L \vdash e_1: \KW{int} \OR \Gamma';U';R';L' \\ 
		\Gamma';\emptyset;R';L' \vdash e_2 : \KW{int} \OR \Gamma'';U'';R'';L''
		}
		{\Gamma;U;R;L \vdash e_1\;op\;e_2: \KW{int} \OR \Gamma'';\emptyset;R'';L''} 
}
\labelRule{\label{trule:inc-alias}}{
	\infrule[Inc-Alias]
		{ \Gamma;U;R;L \vdash e: \KW{int} \OR \Gamma';U';R';L' }
		{\Gamma, x: @\tau'r;U;R;L \vdash x +\!\!= e : \KW{int}\\
		\OR \Gamma', x: @\tau'r;\emptyset;R';L'} 
}

\labelRule{\label{trule:inc}}{
	\infrule[Inc]
		{ \Gamma;U;R;L \vdash e: \KW{int} \OR \Gamma';U';R';L' }
		{\Gamma, x: \text{*}\tau;U;R;L \vdash x +\!\!= e : \KW{int}\\
		\OR \Gamma', x: \text{*}\tau;\emptyset;R';L'} 
}

\labelRule{\label{trule:dec-alias}}{
	\infrule[Dec-Alias]
		{ \Gamma;U;R;L \vdash e: \KW{int} \OR \Gamma';U';R';L' }
		{\Gamma, x: @\tau'r;U;R;L \vdash x -\!\!= e : \KW{int}\\
		\OR \Gamma', x: @\tau'r;\emptyset;R';L'} 
}

\labelRule{\label{trule:dec}}{
	\infrule[Dec]
		{ \Gamma;U;R;L \vdash e: \KW{int} \OR \Gamma';U';R';L' }
		{\Gamma, x: \text{*}\tau;U;R;L \vdash x -\!\!= e : \KW{int}\\
		\OR \Gamma', x: \text{*}\tau;\emptyset;R';L'} 
}

\labelRule{\label{trule:not}}{
	\infrule[Not]
		{\Gamma;U;R;L \vdash e: \KW{int} \OR \Gamma';U';R';L'}
		{\Gamma;U;R;L \vdash !e : \KW{int} \OR \Gamma';\emptyset;R';L'} 
}

\labelRule{\label{trule:deref-ptr}}{
	\infrule[Deref-Ptr]
		{\Gamma;U;R;L \vdash e : \text{*}{!\tau} \OR \Gamma';U';R';L'}
		{\Gamma;U;R;L \vdash \text{*}e : !\tau \OR \Gamma' \cup U';\emptyset;R';L'} 
}

\labelRule{\label{trule:deref-alias}}{
	\infrule[Deref-Alias]
		{\Gamma;U;R;L \vdash e : @!\tau'r \OR \Gamma';U';R';L' \quad \FN{live}('r)}
		{\Gamma;U;R;L \vdash \text{*}e : !\tau \OR \Gamma' \cup U';\emptyset;R';L'} 
}

\labelRule{\label{trule:alias}}{
	\infrule[Alias]
		{'r = \text{newlabel}}
		{\Gamma, x:\text{*}\tau;U;R;L \vdash \KW{alias}\;x : @\tau'r \OR \Gamma,x:\text{*}\tau;U;R;L,('r, \{x\})} 
}

\labelRule{\label{trule:alias-deref}}{
	\infrule[Alias-Deref]
		{\FN{live}('r)}
		{\Gamma, x:@\text{*}\tau'r;U;R;L \vdash \KW{alias}\text{*}\;x : @\tau'r \OR \Gamma,x: @\text{*}\tau'r;U;R;L} 
}

\labelRule{\label{trule:compose}}{
	\infrule[Compose]
		{\Gamma;U;R;L \vdash e_1 :\,!\tau_1 \OR \Gamma';U';R';L' \\ \Gamma';\emptyset;R';L' \vdash e_2 : \tau_2 \OR \Gamma'';U'';R'';L''}{\Gamma;U;R;L \vdash e_1;e_2 : \tau_2 \OR \Gamma'';U'';R'';L''} 
}

\labelRule{\label{trule:scope}}{
	\infrule[Scope]
		{\Gamma;U;R;L \vdash e : \tau \OR \Gamma';U';R';L'\\
		\Gamma_l = \Gamma'\backslash\Gamma \quad !\Gamma_l}
		{\Gamma;U;R;L \vdash \{e\}:\tau \OR \Gamma' \cap \Gamma;\emptyset;R;\\
		\{(r, l^s) \OR \forall (x, \tau) \in \Gamma_l, \forall (r, l^s), x \notin l^s\} } 
}

\labelRule{\label{trule:new}}{
	\infrule[New]
		{\Gamma;U;R;L \vdash e : \KW{int} \OR \Gamma';U';R';L'}
		{\Gamma;U;R;L \vdash \KW{let}\;x:\text{*}\!\tau = \KW{new}(e):\KW{int} \OR \Gamma', x: \text{*}\!\tau;\emptyset;R';L'}
}

% acho que tem que ser só o delete var, pq dá pra fazer *(&tau)
% não dá, essa operação fica *(@*T) que é inválido via as regras que tu escreveu.

\labelRule{\label{trule:delete}}{
	\infrule[Delete]
		{\Gamma;U;R;L \vdash e : \KW{int} \OR \Gamma';U';R';L'}
		{\Gamma,x:\text{*}\tau;U;R;L \vdash 
		\KW{delete}(x, e): \KW{int} \OR \Gamma';\emptyset;R'; \FN{rmLoan}(x, L')}
}

\labelRule{\label{trule:var-alias}}{
	\infrule[Var-Alias]
		{\Gamma;U;R;L \vdash e: @\tau'r \OR \Gamma';U';R';L'}
		{\Gamma;U;R;L \vdash \KW{var}\;x : @\tau'w := e: \KW{int} \OR \Gamma', x: \tau;\emptyset;R', ('r, 'w);L'}
}

\labelRule{\label{trule:var}}{
	\infrule[Var]
		{\Gamma;U;R;L \vdash e: \tau \OR \Gamma';U';R';L'}
		{\Gamma;U;R;L \vdash \KW{var}\;x : \tau := e: \KW{int} \OR \Gamma', x: \tau;\emptyset;R';L'} 
}

\labelRule{\label{trule:assing}}{
	\infrule[Assign]
		{\Gamma;U;R;L \vdash e_1:\,! \tau \OR \Gamma';U';R';L' \\
		\Gamma';\emptyset;R';L' \vdash e_2 :\,! \tau \OR \Gamma'';U'';R'';L'' }
		{\Gamma;U;R;L \vdash e_1 := e_2:\,! \tau \OR \Gamma'';\emptyset;R'';L''} 
}

\labelRule{\label{trule:swap}}{
	\infrule[Swap]
		{}
		{\Gamma,x_1:\tau, x_2:\tau;U;R;L \vdash x_1 :=: x_2 : \KW{int} \\
		\OR \Gamma,x_1:\tau, x_2:\tau;\emptyset;R;\FN{rmLoan}(x_1, \FN{rmLoan}(x_2, L))} 
}

\labelRule{\label{trule:swap-deref-ptr}}{
	\infrule[Swap-Deref-Ptr]
		{}
		{\Gamma,x_1:\tau, x_2:\text{*}\tau;U;R;L \vdash x_1 :=:\!\!\text{*} x_2 : \KW{int} \\
		\OR \Gamma,x_1:\tau, x_2:\text{*}\tau;\emptyset;R;\FN{rmLoan}(x_1, \FN{rmLoan}(x_2, L))} 
}

\labelRule{\label{trule:swap-deref-alias}}{
	\infrule[Swap-Deref-Alias]
		{\FN{live}('r)}
		{\Gamma,x_1:\tau, x_2:@\tau'r;U;R;L \vdash x_1 :=:\!\!\text{*} x_2 : \KW{int} \\
		\OR \Gamma,x_1:\tau, x_2:@\tau'r;\emptyset;R;L} 
}

%adicionando as relações de região
\labelRule{\label{trule:if-alias}}{
	\infrule[If-Alias]
		{\Gamma;U;R;L \vdash e_1: \KW{int} \OR \Gamma';U';R';L' \\
		\Gamma';\emptyset;R';L' \vdash \{ e_2 \} : @\tau'r \OR \Gamma''_2;U''_2;R''_2;L''_2\\
		\Gamma';\emptyset;R';L' \vdash \{ e_3 \} : @\tau'w \OR \Gamma''_3;U''_3;R''_3;L''_3
		}
		{\Gamma;U;R;L \vdash \KW{if}(e_1)\;e_2\;\KW{else}\;e_3 : \tau \\
		\OR \Gamma''_2 \cap \Gamma''_3;\emptyset;R'' \cup \{('r, 'w), ('w, 'r)\};L''_2 \cup L''_3} 
}

\labelRule{\label{trule:if}}{
	\infrule[If]
		{\Gamma;U;R;L \vdash e_1: \KW{int} \OR \Gamma';U';R';L' \\
		\Gamma';\emptyset;R';L' \vdash \{ e_2 \} : \tau \OR \Gamma''_2;U''_2;R''_2;L''_2\\
		\Gamma';\emptyset;R';L' \vdash \{ e_3 \} : \tau \OR \Gamma''_3;U''_3;R''_3;L''_3
		}
		{\Gamma;U;R;L \vdash \KW{if}(e_1)\;e_2\;\KW{else}\;e_3 : \tau \\
		\OR \Gamma''_2 \cap \Gamma''_3;\emptyset;R'';L''_2 \cup L''_3} 
}

\labelRule{\label{trule:callfunc}}{
	\infrule[CallFunc]
		{
			\tau = (\tau_1^f, \tau_2^f, ..., \tau_n^f)\to \tau_r\\
			\Gamma;U;R;L \vdash e_1: \tau_1^a \OR \Gamma_1;U_1;R_1;L_1 \quad \tau_1^f = \tau_1^a\\
			\Gamma_1;\emptyset;R_1;L_1 \vdash e_2: \tau_2^a \OR \Gamma_2;U_2;R_2;L_2 \quad \tau_2^f = \tau_2^a\\
			... \\
			\Gamma_{n-1};\emptyset;R_{n-1};L_{n-1} \vdash e_n: \tau_n^a \OR \Gamma_n;U_n;R_n;L_n \quad \tau_n^f = \tau_n^a\\
			RR = \{(r_i, w_i) \OR \forall i, \tau_i^f = @{\tau}{'}{w_i} \land \tau_i^a = @{\tau}{'}{r_i}\}\\
			\forall (r, w) \in RR, \FN{live}(r, L_n)
		}
		{\Gamma, f: \tau;U;R;L \vdash \KW{let}\;x : \tau_r = f(\bar e) : \KW{int}\\ \OR
		\Gamma_n, f: \tau, x: \tau_r ;\emptyset; R_n \cup RR ;L_n} 
}

\labelRule{\label{trule:stop}}{
	\infrule[Stop]
		{}
		{\Gamma;U;R;L \vdash \KW{stop}: \KW{int} \OR \Gamma;U;R;L} 
}

\labelRule{\label{trule:nullptr}}{
	\infrule[NullPtr]
		{}
		{\Gamma;U;R;L \vdash \KW{nullptr}\typearg{\tau}: \text{*}\tau \OR \Gamma;U;R;L} 
}

\labelRule{\label{trule:nullalias}}{
	\infrule[NullAlias]
		{}
		{\Gamma;U;R;L \vdash \KW{nullalias}\typearg{\tau}: @\tau'0 \OR \Gamma;U;R;L} 
}

\labelRule{\label{trule:func-alias}}{
	\infrule[Func-Alias]
		{
			\Gamma, \overline{x_i:\tau_i}, f: (\bar\tau_i) \to @\tau_r'r;U;R;L \vdash \{e\}: @\tau_r'w \OR \Gamma';U';R';L'\\
			\forall ({'r_1}, {'r_2}) \in R', {'r_2} = {'w} \to {'r} = {'w} \lor ('r, {'r_1}) \in R' \land ({'r_1}, {'r}) \notin R'\\
			\Gamma, f: (\bar\tau_i) \to @\tau_r'r;U;R;L \vdash F : \KW{int} 
		}
		{\Gamma;U;R;L \vdash \KW{fn}\;f(\overline{x_i:\tau_i}) \to @\tau_r'r\;e\;F: \KW{int}} 
}

\labelRule{\label{trule:func}}{
	\infrule[Func]
		{
			\Gamma, \overline{x_i:\tau_i}, f: (\bar\tau_i) \to \tau_r;U;R;L \vdash \{e\}: \tau_r \\
			\Gamma, f: (\bar\tau_i) \to \tau_r;U;R;L \vdash F : \KW{int} 
		}
		{\Gamma;U;R;L \vdash \KW{fn}\;f(\overline{x_i:\tau_i}) \to \tau_r\;e\;F: \KW{int}} 
}

As regras \hyperref[trule:delete]{(DELETE)} e \hyperref[trule:scope]{(SCOPE)} realizam a limpeza de valores do ambiente de concessões $L$. A operação de deletar uma variável é mais granular, removendo apenas as regiões que tem essa variável na sua lista de concessões. A operação de escopo remove todas as variáveis locais definidas dentro do escopo, obtidas via $\Gamma'\backslash\Gamma$ (subtração do conjunto de variáveis declaradas antes da avaliação do escopo com o conjunto de variáveis declarada até o final do escopo). Usa-se notação $!\Gamma$, com o $!$, para indicar que todas as variáveis nesse ambiente são não lineares. O resultado do escopo também descarta todas as associações entre regiões criadas dentro dele, restaurando o conjunto de regiões $R$ externo.

As operações que geram tipos ponteiros \hyperref[trule:new]{(NEW)} e \hyperref[trule:callfunc]{(CALLFUNC)} realizam sempre um \emph{bind} a uma variável. Isso serve para evitar o vazamento de memória e manter as regras de tipos consistentes. Essa chamada de função também descreve que cada argumento deve ser igual ao tipo correspondente na função, sendo avaliados em ordem e usando os ambientes gerados pelos anteriores. A última linha das pré-condições coleta todos os tipos que forem \emph{alias} e adiciona relações de sobrevivência entre eles para a avaliação da função. Essas funções são avaliadas na declaração, regra \hyperref[trule:func-alias]{(FUNC-ALIAS)} e caso retornem um \emph{alias}, precisam testar que o \emph{alias} de retorno do corpo só é sobrevivido pelo \emph{alias} de retorno anotado na função. Isso é discorrido na segunda linha da pré-condição, que diz que se a região de retorno do \emph{alias} do corpo é sobrevivida por alguma outra região, essa região deve ser igual à região do \emph{alias} de retorno anotado ou ser sobrevivida por este e não o sobreviver. Nessa regra ainda falta cavar uma exceção para a região estática $'0$.

%   5.4 Definição via compilação
\section{Definição via Compilação}
\label{chap5:compile}

\newcommand{\compf}[1]{\overset{\texttt{fn}}{\underset{\texttt{comp.}}{[\,#1\,]}}}
\newcommand{\compe}[1]{\overset{\texttt{exp}}{\underset{\texttt{comp.}}{[\,#1\,]}}}
\newcommand{\TMP}{{\color{red}tmp}}
\newcommand{\DEF}{{\color{red}\FN{def}({\color{black}\tau})}}

A definição via compilação é relativamente simples. A linguagem foi criada de uma forma que esse processo fosse simples. Para a maioria dos elementos basta apenas apagar os tipos, entretanto para os operadores de troca ($:=:$ e $:=:\!\!\text{*}$) e para o $\KW{new}$ é necessário algumas operações adicionais. Na compilação há duas operações destacadas em vermelho, $\TMP$ e $\DEF$. A primeira, $\TMP$, representa um nome temporário único no sistema, quanto o segundo, $\DEF$ retorna o valor base para o tipo $\tau$. O valor base para um $\KW{int}$ é 0 e para um ponteiro e \emph{alias} é o ponteiro nulo. Na definição da função recursiva, decidiu-se usar uma notação um tanto inconvencional para evitar que a notação que descreve a compilação seja confundida com o alvo de compilação em si. Dessa forma, há duas funções definidas, $\compf{F}$ e $\compe{e}$, compilando funções e expressões respectivamente. O lado esquerdo do $\Rightarrow$ define a entrada e o direito a saída. As definições da compilação estão na figura \ref{fig:def:compilation}.

\begin{figure}[ht]
	\caption{Função recursiva que define a compilação de $PCL_{front}$ para $PCL_{back}$}
	\label{fig:def:compilation}
	\begin{align}
		&\compf{\KW{fn}\;f(\overline{x : \tau})\to \tau_r\;e\;F} \Rightarrow \KW{let}\;f(\bar x)\; \compe{e}\; \compf{F} \\
		&\compf{\KW{fn}\;()\;e} \Rightarrow \KW{let}\;()\; \compe{e}
	\end{align}
	\begin{minipage}{.45\linewidth}
		\begin{align}
			&\compe{\text{*}e} \Rightarrow \text{*}\compe{e} \\
			&\compe{\KW{alias}\;x} \Rightarrow x \\
			&\compe{\KW{alias}\!\text{*}\;x} \Rightarrow \text{*}x \\
			&\compe{e_1;e_2} \Rightarrow \compe{e_1};\compe{e_2} \\
			&\compe{e_1 := e_2} \Rightarrow \compe{e_1} := \compe{e_2} \\
			&\compe{\{e\}} \Rightarrow \{\compe{e}\} 
		\end{align}
	\end{minipage}
	\hspace{.09\linewidth}
	\begin{minipage}{.45\linewidth}
		\begin{align}
			&\compe{e_1 \; op \; e_2} \Rightarrow \compe{e_1} \; op \; \compe{e_2}\\
			&\compe{x +\!\!= e} \Rightarrow x := x + \!\!\compe{e}; 1 \\
			&\compe{x -\!\!= e} \Rightarrow x := x - \!\!\compe{e}; 1 \\
			&\compe{!e} \Rightarrow\;!\compe{e} \\
			&\compe{x} \Rightarrow x \\
			&\compe{\&x} \Rightarrow \&x 
		\end{align}
	\end{minipage}
	\begin{align}
		&\compe{\KW{let}\; x: \text{*}\tau = \KW{new}(e)} \Rightarrow \KW{let}\;\TMP[1]; \TMP := \compe{e} + 1;\KW{let}\;x[1]; x := \KW{malloc}(\TMP);\nonumber\\
		&\quad\quad \KW{while}(\TMP := \TMP - 1; \TMP > 0)\; \text{*}(x + \TMP - 1) := \DEF; 1 \label{fig:def:compilation:new} \\
		&\compe{\KW{delete}(x, e)} \Rightarrow \KW{free}(x,\compe{e}); 1 \\
		&\compe{\KW{var}\; x: \tau = e} \Rightarrow \KW{let}\;x[1]; x := \compe{e}; 1 \\
		&\compe{x_1 :=: x_2} \Rightarrow \KW{let}\;\TMP[1]; \TMP := x_1; x_1 := x_2; x_2 := \TMP; 1 \label{fig:def:compilation:swap}\\
		&\compe{x_1 :=:\!\!\text{*}\,x_2} \Rightarrow \KW{let}\;\TMP[1]; \TMP := x_1; x_1 := \text{*}x_2; \text{*}x_2 := \TMP; 1 \label{fig:def:compilation:swapderef}\\
		&\compe{\KW{if}(e_1)\;e_2\;\KW{else}\;e_3} \Rightarrow if(\compe{e_1})\;\compe{e_2}\;\KW{else}\;\compe{e_3}\\
		&\compe{\KW{let}\; x: \tau = f(\bar e)} \Rightarrow \KW{let}\;x[1]; x := f(\compe{\bar e}); 1 
	\end{align}
	\begin{minipage}{.25\linewidth}
		\begin{align}
			&\compe{\KW{stop}} \Rightarrow \KW{PANIC}
		\end{align}
	\end{minipage}
	\hspace{0.02\linewidth}
	\begin{minipage}{.30\linewidth}
		\begin{align}
			&\compe{\KW{nullptr}\typearg{\tau}} \Rightarrow \KW{NULL} 
		\end{align}
	\end{minipage}
	\hspace{0.02\linewidth}
	\begin{minipage}{.33\linewidth}
		\begin{align}
			\compe{\KW{nullalias}\typearg{\tau}} \Rightarrow \KW{NULL} 
		\end{align}
	\end{minipage}
	\legend{Fonte: Os Autores.}
\end{figure}

As definições de troca \ref{fig:def:compilation:swap} e \ref{fig:def:compilation:swapderef} realizam expressão idiomática de troca, usando a variável de nome auxiliar $\TMP$. A definição do operador $\KW{new}$, \ref{fig:def:compilation:new}, é simples se não pelo uso do  $\KW{while}$ para atribuir valores-base ao espaço alocado. A compilação de $PCL_{front}$ para $PCL_{back}$ ainda precisa ser testada e provada correta.

\section{Implementação}

\todo[inline]{escrever}
