% 4. Use Case: Borrow Checker (explica o borrow checker e o algoritmo do polonius)
\chapter{Borrow Checker}
\label{chap4}
% O que é e o porque de um borrow checker
% One of the most researched solutions for memory bugs has been 
% Rust's borrow checker\cite{RUSTBOOK}. It promises a zero overhead Temporal Safety 
% solution by constraining the ability to alias and adding code annotations 
% to validate temporal memory access. Although, properly validating its claims 
% has been a challenge\cite{RUSTBELT, RUSTSYMBOLIC}.

Com a base $PCL_{back}$ definida, agora se avança para a próxima etapa do projeto, provar alguma solução de memória. Uma das soluções mais estudadas é o \emph{Borrow Checker} de Rust \cite{RUSTBOOK}. Ele promete uma solução para \emph{bugs} temporais com zero custo no desempenho do programa. Para isso, não só se limita a habilidade do programador de realizar \emph{alias}, como se necessita que ele anote o código para validar os acessos temporais. Entretanto, a validação desse sistema todo tem sido um desafio \cite{RUSTBELT,RUSTSYMBOLIC}.

% In the case of $PCL_{back}$, there is an interesting bijection between the key locks 
% used to check for temporal safety and the lifetimes used for computing the correctness 
% of an access. Therefore, one could postulate that to implement a front-end language 
% with a borrow checking system $PCL_{front}$, it could be proven that 
% no \textit{Use After Free} errors would occur in $PCL_{back}$, 
% given a correct input program and a sound compilation.

Esse sistema foi escolhido especificamente porque há uma bijeção interessante entre as chaves e fechaduras usadas no mecanismo de detecção de falhas temporais de $PCL_{back}$ e os tempos de vida usados pelo \emph{Borrow Checker} de Rust para validar os acessos. Dessa forma, pode-se postular que ao implementar uma linguagem de \emph{front-end} com um sistema de \emph{Borrow Checker}, $PCL_{front}$, poder-se-ia provar que nenhum erro de \emph{Use-After-Free} ocorreria no $PCL_{back}$ compilado de $PCL_{front}$ dado um programa de entrada e compilação corretos.


%	4.1 Rust and the Borrow Checker
\section{O Borrow Checker do Rust}

%	4.2 Tipos Lineares
\section{Tipos Lineares}
% Call back da sessão anterior, recontextualizando o comportamento do borrow checker 
% do Rust como tipos lineares/Afim

%	4.3 Polonius
\section{Polonius}
% Explicar os detalhes da implementação do polonius
% Expecífico ao domínio do Rust