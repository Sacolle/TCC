% 4. Use Case: Borrow Checker (explica o borrow checker e o algoritmo do polonius)
\chapter{Borrow Checker}
\label{chap4}
% O que é e o porque de um borrow checker
% One of the most researched solutions for memory bugs has been 
% Rust's borrow checker\cite{RUSTBOOK}. It promises a zero overhead Temporal Safety 
% solution by constraining the ability to alias and adding code annotations 
% to validate temporal memory access. Although, properly validating its claims 
% has been a challenge\cite{RUSTBELT, RUSTSYMBOLIC}.

Com a base $PCL_{back}$ definida, agora se avança para a próxima etapa do projeto, provar alguma solução de memória. Uma das soluções mais estudadas é o \emph{Borrow Checker} de Rust \cite{RUSTBOOK}. Ele promete uma solução para \emph{bugs} temporais com zero custo no desempenho do programa. Para isso, não só se limita a habilidade do programador de realizar \emph{alias}, como se necessita que ele anote o código para validar os acessos temporais. Entretanto, a validação desse sistema todo tem sido um desafio \cite{RUSTBELT,RUSTSYMBOLIC}.

% In the case of $PCL_{back}$, there is an interesting bijection between the key locks 
% used to check for temporal safety and the lifetimes used for computing the correctness 
% of an access. Therefore, one could postulate that to implement a front-end language 
% with a borrow checking system $PCL_{front}$, it could be proven that 
% no \textit{Use After Free} errors would occur in $PCL_{back}$, 
% given a correct input program and a sound compilation.

Esse sistema foi escolhido especificamente porque há uma bijeção interessante entre as chaves e fechaduras usadas no mecanismo de detecção de falhas temporais de $PCL_{back}$ e os tempos de vida usados pelo \emph{Borrow Checker} de Rust para validar os acessos. Dessa forma, pode-se postular que ao implementar uma linguagem de \emph{front-end} com um sistema de \emph{Borrow Checker}, $PCL_{front}$, poder-se-ia provar que nenhum erro de \emph{Use-After-Free} ocorreria no $PCL_{back}$ compilado de $PCL_{front}$ dado um programa de entrada e compilação corretos.

%	4.1 Rust and the Borrow Checker
\section{O Borrow Checker de Rust}

Rust é uma linguagem de programação de baixo nível, comparável a C++, que oferece acesso seguro à memória sem custo ao tempo de execução do programa. Ao invés de utilizar um GC para abstrair o acesso à memória do programador, Rust se utiliza de uma série de regras para checar e gerir, em tempo de compilação, os acessos à memória, evitando erros como \emph{Use-After-Free}, \emph{Double-Free}, etc. O mecanismo dentro da linguagem que faz essa validação é o \emph{Borrow Checker}. 

A intuição do funcionamento do \emph{Borrow Checker} parte do entendimento de como são geridas alocações na pilha e na \emph{heap}. Um caso de estudo é, por exemplo, uma \emph{string} dinâmica $s1$, declarada como \lstinline[language=Rust]|let s1 = String::from("hello")|. Ela possui um componente da pilha, a tripla ponteiro, tamanho, capacidade, com o ponteiro referenciando o componente da \emph{heap}, que guarda os caracteres em si. O diagrama \ref{fig:svg:mem-layout} demonstra o \emph{layout} dessa estrutura em memória.

\begin{figure}[ht]
	\centering
	\caption{\emph{Layout} de memória de uma \emph{string} dinâmica em Rust.}
	\label{fig:svg:mem-layout}
	\includesvg[scale=0.35]{figures/string-layout.svg}
	\legend{Fonte: \citet[Figure 4-1]{RUSTBOOK}}
\end{figure}

Estruturas na pilha são trivialmente copiadas, visto que o tamanho delas é sempre conhecido. Elementos na \emph{heap} podem representar diversas estruturas complexas sem tamanho definido.  Dessa forma, tipos em Rust podem \lstinline[language=Rust]|Copy| ou \lstinline[language=Rust]|Clone|, representando, respectivamente, a cópia automática de valores, para estruturas compostas apenas por elementos na pilha, e a possibilidade de cópia manual, para estruturas que possuem uma parte na \emph{heap}. A interação que faz esse conceito progredir é que operações de atribuição, tal como \lstinline[language=Rust]|a = b|, só copiam o componente na pilha de um valor. Por consequência, valores que não são \lstinline[language=Rust]|Copy| adquirem um novo contexto semântico na atribuição, em que ao invés do elemento da esquerda receber o valor do elemento da direita, o elemento da direita é movido para o elemento da esquerda. 

Esse conceito de mover valores introduz a ideia de \emph{ownership}. Nela, variáveis são donas de valores e em certas operações (atribuição, passagem de parâmetros, capturas...), caso o tipo não seja \lstinline[language=Rust]|Copy|, a propriedade do valor é passada de um elemento a outro. O acesso a um valor só pode ser feito então pelo elemento que é dono deste. No trecho de código \ref{fig:ex:move}, $s1$ é dono do valor \lstinline[language=Rust]|String::from("hello")| e este é \textbf{movido} para $s2$. O acesso então a $s1$ após mover o valor passa a ser inválido, pois ele não é mais o dono do valor.

\begin{figure}[ht]
	\caption{Exemplo de \emph{move} em Rust}
	\label{fig:ex:move}
	\begin{lstlisting}[language=Rust]
let s1 = String::from("hello");
let s2 = s1;
s1.len(); //erro, uso de valor movido.
	\end{lstlisting}
	\legend{Fonte: Os Autores.}
\end{figure}

Outro aspecto importante é que todos os valores têm a sua função \lstinline[language=Rust]|drop| chamada quando os seus donos saem de escopo, visto que é apenas através deles que o valor pode ser acessado, não é necessário manter o objeto em memória se não há nenhum elemento vivo para referenciá-lo. Com isso, introduz-se o conceito do \emph{Borrow} (empréstimo). Um empréstimo em Rust nada mais é que um ponteiro não anulável a um valor, tal qual referências em C++. Esse empréstimo pode ser imutável, podendo existir várias cópias dele, ou mutável, podendo existir só um. Esses modos são mutuamente exclusivos, ou um empréstimo mutável, ou vários empréstimos imutáveis. Eles também carregam uma notação que indica o seu tempo de vida, denotado por uma breve \emph{string} de caracteres prefixado com uma aspa simples, tal como \lstinline[language=Rust]|'a|.

O acesso a esses empréstimos é validado, em tempo de compilação, com o uso de \emph{Borrow Checker}, que analisa os tempos de vidas deles e verifica se todos os acessos são a elementos vivos no escopo de execução (vivo aqui denota que ele ainda não saiu de escopo ou foi manualmente liberado). Há várias estratégias para validar esses acessos. A atual é uma mistura de NLL \cite{NLL} e Polonius \cite{Stjerna1684081}, que será elaborado em sequência.

%Essa dicotomia está no cerne do \emph{Borrow Checker}. 
O \emph{Borrow Checker} então pode ser definido como uma ferramenta para validar os acessos a ponteiros, construído sobre uma base que limita a criação desses diretamente para estruturas na \emph{heap}. Pode-se observar muita semelhança com o sistema de \emph{move} de Rust para tipos não \lstinline[language=Rust]|Copy| com o \emph{unique\_ptr} de C++. Os dois impedem o \emph{alias} múltiplo, liberando a sua memória no final do escopo. 

%	4.2 Tipos Lineares
\section{Tipos Lineares}

Tipos Lineares é um sistema de tipos derivado da lógica linear intuicionista \cite{CSLINLOG}. Nele, valores de tipos lineares devem ser usados exatamente uma vez. A pesquisa desse sistema de tipos se iniciou principalmente focada na capacidade dele de reduzir alocações em linguagens funcionais \cite{Wadler1990LinearTC}. A tendência de muitos desses sistemas era de implementar tipos lineares junto de tipos não lineares. Dessa forma, alguns elementos da linguagem, os não lineares, podem ser usados várias vezes (ou não usados), e outros, os lineares, podem ser usados só uma vez. 

Essa dinâmica entre um sistema com tipos lineares e não lineares evoca fortemente o sistema de tipos de Rust, com tipos \lstinline[language=Rust]|Clone| e \lstinline[language=Rust]|Copy|. Os tipos não \lstinline[language=Rust]|Copy| de Rust podem ser modelados como tipos lineares. O uso de um valor em Rust é o ato de mover ele para outro dono: o símbolo que tinha o valor não o carrega mais, foi usado. Além disso, pode-se pensar que todos os valores são usados pelas suas respectivas funções \lstinline[language=Rust]|drop()| no final de seus escopos. A linguagem Rust pode então ser dita quase linear, tendo muita sobreposição com outros mecanismos quase lineares, como o \lstinline[language=C]|unique_ptr| de C++. 

%complementar?
%	4.3 Polonius
\section{Polonius}
\label{sec:chap4:polonius}

% Explicar os detalhes da implementação do polonius
% Expecífico ao domínio do Rust

O algoritmo Polonius é uma implementação do \emph{Borrow Checker} que interpreta os tempos de vidas como regiões e os empréstimos como concessões, que podem ser revogadas. Assim, um empréstimo \lstinline[language=Rust]|&'a T| possui a região $'a$. Ela é associada a um conjunto de concessões, tal que invalidar os termos de qualquer concessão no conjunto de $'a$, invalida o empréstimo $r$ a que se associa. Invalidar os termos da concessão significa: ``Realizar um acesso ilegal ao caminho emprestado pela concessão.'' \cite{Matsakis2018}. Ou seja, se uma variável possui um empréstimo ativo e é acessada diretamente, isso invalida esse empréstimo ao realizar um acesso ilegal no caminho dele, sendo esse caminho a variável em si como em \ref{fig:ex:polonius}. 
\begin{figure}[ht]
	\caption{Exemplo de acesso ilegal ao caminho de um empréstimo.}
	\label{fig:ex:polonius}
	\begin{lstlisting}[language=Rust]
let s = String::from("hello");
let emp_s: &'a String = &'b s; //|.{\color{codegreen}empréstimo realizado a s}.
//|.{\color{codegreen}move s para a função operation}.
//|.{\color{codegreen}isso representa um acesso ao caminho de emp\_s, o acesso a s em sí}.
operation(a); 
//|.{\color{codegreen}a partir daqui o acesso a emp\_s é inválido}.
	\end{lstlisting}
	\legend{Fonte: Os Autores.}
\end{figure}

As regiões de Polonius também definem regras de subtipagem $<:$ entre si. Dado a linha 2 de \ref{fig:ex:polonius}, há uma relação de subtipo entre $\&'a String$ e $\&'b String$ (s é do tipo \lstinline[language=Rust]|String|), podendo ser expressa como:
\[ \frac{'b \subseteq {'a}}{\&'b\;String <: \&'a\;String} \]

A parte relevante é a pré-condição, $'b \subseteq {'a}$, ela é mais comumente expressa como $'b : {'a}$, em que se lê $b$ \emph{outlives} (sobrevive) $a$. Essas relações de sobrevivência são transitivas em um programa, da forma $'b : {'a} \land {'a} : {'c} \to {'b} : {'c}$. Todas as regiões sobrevivem a si mesmas, ${'b} : {'b}$ e são sobrevividas pela região estática $'static$. Pode-se formular essa relação da seguinte forma: dado todas as regiões de um programa $r_1,r_2 \in Reg$, define-se a relação de sobrevivência como o fechamento binário $r_1\;R\;r_2$ sobre $Reg$, sendo $R$ transitivo e tendo a propriedade $\forall (r_1, r_2) \in R, (r_1, r_1) \in R \land (r_2, r_2) \in R$.

Essas regiões $r \in Regs$ são associadas a um conjunto de empréstimos $l^s \subseteq Loans$ do programa no fechamento binário $r\;L\;l^s$. Essa relação é criada no processo de criação de empréstimo usando o operador $\&$. Um exemplo disso é a linha 2 de \ref{fig:ex:polonius}, nela, cria-se a relação $('b, \{ \&x \})$, em que descreve que a região $'b$ tem uma referência imutável a $x$ no seu conjunto de concessões. Essas concessões $l \in Loans$ propagam-se sobre as relações de $R$ em $L$. Desse jeito, todas as concessões em $'b$ exitem em todos os elementos que ele sobrevive, sendo descrito na forma:
\[ \forall ('b, 'a) \in R, \exists ('c, ls_1), ('d, ls_2) \in L, {'b} = {'c} \land {'a} = {'d} \to ls_1 \subseteq ls_2 \]

Com essa propagação, quando uma concessão tem seus termos invalidados, ela inválida todas as regiões que a tem associada. Um exemplo de um caso que é valido é \ref{fig:ex:polonius:1}, em que mesmo que o empréstimo tenha sido feito dentro do escopo, devido à propagação, esse valor é vivo fora deste. Um exemplo de um caso que é inválido é \ref{fig:ex:polonius:2}, em que o empréstimo é invalidado no final do escopo, propagando esse fato e invalidando o acesso ao empréstimo.

\begin{figure}[ht]
	\caption{Exemplo de acesso válido usando Polonius.}
	\label{fig:ex:polonius:1}
	\begin{lstlisting}[language=Rust]
let r : &'a i32;
let num : i32 = 12;
{
	// |.{\color{codegreen}$('1, 'a) \in R$, $('1, \{ \&\text{num} \}) \in L$}.
	//|.{\color{codegreen}pelas relações acima, tem-se também que $('a, \{ \&\text{num} \}) \in L$}.
	r = &'1 num; 
}
//|.{\color{codegreen}o acesso a r é valido aqui, pois a região 'a está viva $('a, \{ \&num \}) \in L$}
	\end{lstlisting}
	\legend{Fonte: Os Autores}
\end{figure}


\begin{figure}[ht]
	\caption{Exemplo de acesso inválido usando Polonius. }
	\label{fig:ex:polonius:2}
	\begin{lstlisting}[language=Rust]
let r : &'a i32;
{
	let num : i32 = 12;
	//|.{\color{codegreen}$('1, {'a}) \in R$, $('1, \{ \&\text{num} \}) \in L$}.
	//|.{\color{codegreen}pelas relações acima, tem-se também que $('a, \{ \&\text{num} \}) \in L$}.
	r = &'1 num; 
	//|.{\color{codegreen}Aqui é feito o drop de num, então invalida-se a concessão \&num}.
}
//|.{\color{codegreen}o acesso a r é invalido aqui, pois a 'a está não está viva $\forall ls \subseteq Loans,('a, ls) \notin L$}

	\end{lstlisting}
	\legend{Fonte: Os Autores}
\end{figure}


Alguns elementos importantes de Polonius foram omitidos. Sua implementação ocorre sobre o \emph{Control Flow Graph} (CFG) de um programa Rust, sendo executadas via programa Datalog \cite{DATALOG}, implementado em Rust via Datafrog \cite{DATAFROG}. Essa avaliação é mais rica do que a descrita, pois precisa localizar mais fortemente a posição do erro de \emph{Borrow Checker} para informar ao programador. Mesmo assim, as regras acima descrevem bem o funcionamento do Polonius, dando o foco macro necessário para como o \emph{Borrow Checker} de $PCL_{front}$ será implementado.

