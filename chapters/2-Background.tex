% 2. Background
\chapter{Conceitos Básicos} 
\label{chap2}

Este capítulo introduz alguns conceitos e definições essenciais ao projeto. 
Outros conceitos, específicos ao domínio do caso de uso da linguagem de \emph{front-end},
são elaborados no Capítulo \ref{chap4}. 

%	2.1 O que é Semantica Operacional
\section{Semântica Operacional e Sistema de Tipos}

Semântica formal diz respeito à descrição matemática e precisa do comportamento dos programas de uma dada linguagem de programação, os quais são descritos por meio de uma sintaxe formal. Ela pode ser dividida em \emph{Natural Semantics} (Semântica Natural) e em \emph{Structural Operacional Semantics} (Semântica Operacional Estrutural). Esses dois modos também são conhecidos como semântica \emph{Big Step} e semântica \emph{Small Step}, respectivamente \cite{NIELSON}. A linguagem $PCL_{back}$ é definida via semântica operacional \emph{Small Step}, que descreve o resultado de cada passo da transição de um estado da avaliação para o outro.

% O significado dos programas é definido por um sistema de transições. No caso da semântica \emph{Small Step}, utilizada na definição das linguagens deste trabalho, esse sistema de transições que leva um programa $P$ e um estado $s$ para um novo programa $P'$ e um novo estado $s'$, da forma $\St{P, s} \to \St{P', s'}$. Essas transições são definidas pela estrutura do programa $P$. Pode-se nomear cada estado possível do programa $P$ como $\gamma$. Alguns desses $\gamma$ são ditos estados terminais, por simbolizarem o fim da computação. Por outro lado, podem existir $\gamma$ que não possuem transição na semântica, ditos que estão \emph{stuck} (presos). Uma execução de um programa $P$ é uma sequência de derivações de estados $\gamma$ finitas, até que atinja um estado terminal ou preso. Essa sequência também pode ser infinita, sem nunca chegar nos estados que param a avaliação. Um fecho que executa $i$ transições pode ser descrito como $\gamma_0 \to^i \gamma_i$. Mais comumente, utiliza-se a notação $\gamma_0 \to^* \gamma_k$ para indicar que em 0 ou mais passos foi-se de uma configuração $\gamma_0$ para uma configuração $\gamma_k$. Assim, pode-se anotar a execução finita de um programa $P$, representado pelo estado inicial $\gamma_0$, para um estado final ou preso $\gamma_f$, como $\gamma_0 \to^* \gamma_f$.

Tipos podem ser utilizados para classificar programas de forma estática e garantir a ausência de certos comportamentos indesejados. Tipicamente, em uma dada linguagem fortemente tipada, somente programas que possuem tipos associados são efetivamente compilados ou interpretados. A associação de programas a tipos é formalmente especificada por meio de julgamentos de tipos, relações formais normalmente especificadas por um conjunto finito de regras. %Essas regras são descritas como um conjunto de regras de inferência que atribuem tipos a termos. 


%   2.2 Erros de memória
\section{Falhas de Memória}
\label{sec:mem-error}

Uma falha de memória é composta de duas partes, a natureza da falha e o ambiente de memória. A parte falha advém da definição da estrutura hierárquica fracasso, erro, falha, descrita em \citet{FAULTS}: 
\begin{quotation}\textit{
	(...) um fracasso de serviço significa que pelo menos um (ou mais) estados externos do sistema desviaram do estado correto do serviço. Esse desvio é chamado de erro. A julgada ou hipotetizada causa de um erro é chamada de falha.}

	\hspace{1em plus 1fill}---\citet[p. 13, tradução nossa]{FAULTS}
\end{quotation}

A parte do ambiente de memória é o espaço do programa em que ocorrem alocações de dados, podendo ser na pilha ou na \emph{heap}. Uma falha de memória ocorre, então, quando um acesso à memória pode levar a execução do programa a um estado indefinido, tal evento é chamado comportamento indefinido (UB). Essas falhas, ou no coloquial \emph{bugs}, podem ser ativas, gerando um erro e movendo o programa para um estado incorreto; ou dormentes, existindo como possíveis vetores de ataque para agentes maliciosos capazes de ativar essa falha. 

As falhas de memória podem ser classificadas em 5 classes: falha Espacial, Temporal, de Tipo, de Inicialização e de Condição de Corrida \cite{Apple22,Google24}. Falhas de memória podem ser subdivididos de mais formas \cite{7KINGDOMS,CWELIST}, tendo em conta falhas mais específicas ao domínio do programa. O uso das classes é fundamental para o domínio deste projeto: cada uma delas mapeia uma série de problemas para soluções específicas e disjuntas das demais.  Ou seja, a solução para os problemas advindos de cada uma dessas classes pode ser pensada independentemente. A seguir serão detalhados alguns tipos particulares de falhas de memória, classificadas entre as 5 classes, indicando quais destas serão efetivamente modeladas pelo trabalho.

\subsection{Falhas Espaciais}
\label{sec:mem-error:spacial}

A classe de falhas espaciais ocorre quando há manipulação de memória a uma região além do espaço alocado. Erros associados a acessos de memória em \emph{buffers} podem ser classificados amplamente como \emph{Out-of-Bounds-Read} ou \emph{Out-of-Bounds-Write} (leitura ou escrita fora dos limites). Podendo ocorrer na pilha ou na \emph{heap}, para o final ou para o começo do \emph{buffer}, devido a um cálculo incorreto de tamanho do \emph{buffer} ou de índice de acesso.  Fundamentalmente, esses erros são associados ao acesso arbitrário a memória e à falta de validação adequada.

Um exemplo dessa falha é um \emph{Buffer-Under-Read}, 
leitura de um \emph{buffer} para uma posição anterior ao começo. No Programa \ref{lst:spacial-error-c}
não se checa se $idx \ge 0$, possivelmente acessando memória antes do ponteiro.

\begin{lstlisting}[language=C ,label={lst:spacial-error-c}, caption=Exemplo de uma Falha Espacial]
int get_val(int* list, int len, int idx){
	if(idx < len) return list[idx];
 	else return -1;
}
\end{lstlisting}

Falhas espaciais podem ser mitigadas de duas grandes formas. Há os tipos dependentes de Checked C \cite{CHECKEDC}, que verificam em tempo de compilação, usando informações de tipo, se um acesso está nos limites do \emph{buffer}. Essa estratégia é menos comum, porém não adiciona nenhum impacto no desempenho do código. 
O método mais adotado atualmente é o de checagem de limite em tempo de execução. Nele, adiciona-se junto ao ponteiro a informação de tamanho do \emph{buffer} apontado, criando um \emph{fat-pointer} (ponteiro gordo). Assim, a cada acesso, verifica-se se este está dentro dos limites do \emph{buffer}, se não estiver, gera uma terminação controlada. Linguagens como Rust, Zig e Go se utilizam desse mecanismo, entretanto sob o nome de \emph{slices}.


\subsection{Falhas Temporais}
\label{sec:mem-error:temporal}

A classe de falhas temporais ocorre quando há acesso a uma região de memória que não está mais ativa no momento. Ela engloba casos de \emph{Use-After-Free} e \emph{Use-After-Return}, assim como erros associados a função \lstinline[language=C]|free| de C. A falha de \emph{Use-After-Free} é todo e qualquer acesso a \emph{heap} quando o objeto que se encontrava no local já havia sido liberado. Se essa falha ocorre na pilha ao invés da \emph{heap}, é chamada de \emph{Use-After-Return}, sendo o acesso a um endereço na pilha que já foi liberado pelo desempilhamento desta.

Um exemplo da falha de \emph{Use-After-Free} é o \autoref{lst:temporal-error-heap-c}. Nele, num caminho de erro do código, libera-se o ponteiro $ptr$. Entretanto, no momento de arquivar esse erro, utiliza-se esse mesmo ponteiro liberado, gerando uma falha que pode escalar a erro dependendo do contexto de execução:
\begin{lstlisting}[language=C, label={lst:temporal-error-heap-c}, caption=Exemplo de uma Falha Temporal na \emph{Heap}]
char* ptr = (char*)malloc (SIZE);  
if (err) {
	abrt = 1;  
	free(ptr);
}  
...  
if (abrt) {
	logError("operation aborted before commit", ptr);
}
\end{lstlisting}

Outro exemplo é o Programa \ref{lst:temporal-error-stack-c}, que retorna um \emph{buffer} alocado na pilha. Acesso a esse elemento pode reescrever a pilha de outras funções, sendo uma falha grave.

\begin{lstlisting}[language=C, label={lst:temporal-error-stack-c}, caption=Exemplo de uma Falha Temporal na Pilha]
char* getName() {
   char name[STR_MAX];  
   fillInName(name);  
   return name;
}
\end{lstlisting}

Fundamentalmente, a classe lida com leitura e escrita a regiões de memória não estáticas, ou seja, alocações na pilha, removidas no fim do escopo; e alocações na memória, que podem ser arbitrariamente liberadas.


% Garbage Collection (GC)
\phantomsection
\label{sec:mem-error:GC}
A mitigação dessa classe de erros é a mais ampla e complexa deste trabalho. Desde a sua invenção em 1959 \cite{GCSTART}, uma das soluções mais completas presente tem sido o uso de um GC (coletor de lixo). Esse mecanismo detecta e retorna automaticamente memória alocada e não mais utilizada para o OS (sistema operacional). Em troca, o programa sofre breves pausa periódicas (GC \emph{pauses}) na execução e a gerência do GC custa processamento que certos ambientes não podem pagar.

% Memory regions\cite{REGMEM}
\phantomsection
\label{sec:mem-error:MemReg}
Outra estratégia que emergiu do mundo de linguagens funcionais foi a de regiões de memória \cite{REGMEM}. Elas foram propostas como uma alternativa mais eficiente e previsível de gerência de memória que GC para MLKit. Nela, realizam-se as alocações em regiões, fazendo a gerência das regiões ao invés das alocações individuais. O projeto Cyclone trouxe esse conceito de regiões para linguagens imperativas no começo do milênio \cite{CYCLONEMEM}, adicionando diversos mecanismos para adequar essa gerência ao novo meio.

%key-lock systems \cite{FATPOINTERS, CCURED},
\phantomsection
\label{sec:mem-error:KeyLock}
Um contemporâneo de Cyclone, CCured \cite{CCURED}, utilizou-se de um sistema de \emph{key-lock} para evitar falhas temporais. Nele, mantém-se uma tabela global com uma chave para cada alocação. Realiza-se uma verificação, na hora do acesso, se a chave ainda é válida; caso não seja, há uma terminação controlada. Esse método se demonstrou muito ineficiente. Mais recentemente, um projeto adicionou um sistema similar usando \emph{fat-pointers} em Checked C \cite{FATPOINTERS}, mantendo esses metadados mais próximos dos dados reais e aumentando significativamente o desempenho.

%borrow checker
\phantomsection
\label{sec:mem-error:BorrowChecker}
Um método alternativo, com gerência automática de memória, é o \emph{borrow checker} de Rust. Nele, limita-se o \emph{aliasing} de ponteiros e adicionam-se notações para que o acesso à memória possa ser validado em tempo de compilação. O Capítulo \ref{chap4} explorará em maior detalhe os elementos de interesse desse sistema. 

\newcommand{\FREE}{\lstinline[language=C]|free()| }
\phantomsection
\label{sec:mem-error:temporal:free}

\textbf{Free}: Um componente importante dessa classe são as falhas associadas à liberação de memória, principalmente tratando-se da função \FREE da \emph{stdlib} de C. Falhas como \emph{Double-Free} (chamar \FREE para um endereço liberado), \emph{Free-of-Memory-not-on-the-Heap} (chamar \FREE em um ponteiro que não aponta para a \emph{heap}), \emph{Free-of-Pointer-not-at-Start-of-Buffer} (chamar \FREE com um ponteiro que não inicia a lista) e \emph{Release-of-Invalid-Pointer-or-Reference} (chamar uma função de liberação de memória para memória alocada com outro sistema) ocorrem devido a uma ação que corrompe a estrutura de dados usada para gerir a alocação. Assim, alocações subsequentes podem gerar resultados imprevisíveis (UB) e podem escalar a erros.

Um grande componente de soluções de memória temporal como o \hyperref[sec:mem-error:GC]{GC}, as \hyperref[sec:mem-error:MemReg]{Regiões de Memórias} e o \hyperref[sec:mem-error:BorrowChecker]{\emph{Borrow Checker}} é a dispensa da liberação manual de memória. Isso diminui a superfície de falha do código, evitando possíveis erros.

\phantomsection
\label{sec:mem-error:temporal:null}

\textbf{Nullprt}: Acessar o ponteiro nulo é um caso interessante nessa taxonomia de classes. Não se pode dizer que essa ação é, firmemente, uma falha de inicialização (\autoref{sec:mem-error:init}) ou uma falha temporal (\autoref{sec:mem-error:temporal}). Afinal, o valor nunca é liberado ou inicializado em função do tempo. Entretanto, a passagem do ponteiro nulo à função \FREE gera o erro específico de \emph{Free-of-Memory-not-on-the-Heap}, então justifica-se discorrer junto às falhas temporais. O acesso ao ponteiro nulo também é facilmente detectável pelo sistema operacional. Dessa forma, quando esta situação ocorre, o sistema operacional tende a interromper o programa imediatamente. Entretanto, em certos sistemas, essa falha pode escalar para ataques mais severos \cite[p.4]{MemErrorPastPresentFuture}.

O ponteiro nulo é um elemento comum da maioria das linguagens, via a influência da linguagem C. Entretanto, para lidar com referências não é necessário o seu uso. Linguagens funcionais, e as que foram inspiradas por elas - como Rust, utilizam-se de tipos somatórios (\emph{tagged unions, variants, choice types, ...}) para representar a ausência de um valor, mantendo as referências não anuláveis.


\subsection{Falhas de Inicialização}
\label{sec:mem-error:init}

A classe de falhas de inicialização ocorre quando há acesso a uma região de memória alocada, mas um valor ainda não foi atribuído a ela, ou seja, a região ainda não foi inicializada. Nesse caso, a leitura do espaço não inicializado gera valores incoerentes. %No caso de ponteiros, pode-se acessar locais de memória arbitrários.

Erros de inicialização podem existir em contextos estáticos ou dinâmicos. No caso de variáveis, somatórios (\emph{tuples, structs, records, ...}) e listas de tamanho fixo, pode-se avaliar a inicialização do campo com uma análise estática de árvore. Para listas de tamanho dinâmico, não há como realizar essa avaliação a tempo de compilação, sendo preferido que a API (interface programável de aplicação) da linguagem requeira que a lista seja inicializada com valores-base para os campos, evitando acesso a campos não inicializados. Isso também se estende às estruturas estáticas compostas: a inicialização e uso parcial delas pode gerar erros e dificultar a respectiva análise. Obrigar o programador a inicializar toda a estrutura não parece ser restritivo e incentiva boas práticas de código.

Um exemplo dessa falha é o acesso a \emph{string} em um caminho que não inicializa ela, representada no Programa \ref{lst:initialization}, que lerá valores arbitrários na \emph{heap}.

\begin{lstlisting}[language=C, label={lst:initialization}, caption=Exemplo de uma Falha de Inicalização com \emph{Strings}]
char *test_string;
if (i != err_val){
	test_string = "Hello World!";
}
printf("%s", test_string);
\end{lstlisting}


\subsection{Falhas de Tipo}

A classe de falhas de tipo ocorre quando há acessos a uma região de memória com o tipo incorreto. Ela ocorre no caso de \emph{casts} incorretos de \emph{pointers} tanto para tipos de tamanhos diferentes, como com \emph{layouts} diferentes. Essa classe de erros é interessante, pois é nela em que há a maior intersecção com \emph{undefined behavior}. \emph{Undefined behaviors} (UBs) variam conforme a especificação de cada linguagem, e não será um tópico explorado em detalhe neste texto. Mas, brevemente, o erro no caso do \emph{cast} pode ser de ler a memória incorretamente, mas também pode ser por alguma otimização do compilador, que ou mudou o \emph{layout} do objeto de origem/destino, ou alterou o resultado da operação, pois a ação gerava um comportamento indefinido (UB).

\subsection{Falhas de Condição de Corrida}

A classe de falhas de condição de corrida compreende todos os problemas gerados com código concorrente/paralelo. Condições de corrida são extremamente importantes em programas paralelos e concorrentes, e são causas de muitas falhas de difícil detecção. Este trabalho foca nas falhas presentes em modelos sequenciais de computação, portanto essa classe particular de problemas não será desenvolvida. Caso o leitor tenha interesse sobre o tema, recomenda-se o livro \emph{An Introduction to Parallel Programming}  \cite{pacheco2011}.
