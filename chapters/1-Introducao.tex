% 1. Introdução
\chapter{Introdução}

% The C programming language\cite{CLANG} is the foundation of systems programming. 
% Up to the adoption of Rust in early 2024\cite{RUSTFORLINUX}, 
% it was the (only) language of the Linux kernel. 
% Despite this great importance, C is still prone to many types of memory safety violations, 
% with very little guarantees from the compiler.

% At the start of the millennium, Cyclone\cite{CYCLONE1} started the trend of low-level
% programming languages with safe memory systems without the broad use of 
% Garbage Collection (GC). 
% This was followed by CCured\cite{CCURED}, Rust\cite{RUSTBOOK} and more, 
% each with its own memory safety solution. 

%	1.1 Motivação
\section{Motivação}

% Given a memory safety solution, it is important for it to be formally proven correct, 
% as was done by \citet{RUSTBELT} and \citet{RUSTSYMBOLIC} for Rust's borrow checker.
% With that, this work introduces Porcelain: a Semantic Framework for Representing 
% and Analyzing Memory Safety Techniques. 
% The aim is to define a front-end language by compilation to porcelain's defined 
% back-end language as a means to prove certain theorems about the front-end languages 
% memory safety mechanisms.


%	1.2 Related Work
\section{Trabalhos Relacionado}

%	1.3 Objetivos
\section{Objetivos}

%	1.4 Organização
\section{Organização}

%This project in its current state includes a back-end language defined
% by a small-step operational semantics as well as a front-end language,
% modeling a borrow checker, defined via compilation to the back-end language.

\todo[inline]{Não tem necessidade de fazer essa sessão agora, muito menos o capítulo.}