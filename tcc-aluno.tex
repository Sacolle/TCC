%
% exemplo genérico de uso da classe iiufrgs.cls
% $Id: iiufrgs.tex,v 1.1.1.1 2005/01/18 23:54:42 avila Exp $
%
% This is an example file and is hereby explicitly put in the
% public domain.
%
\documentclass[cic,tc]{iiufrgs}
% Para usar o modelo, deve-se informar o programa e o tipo de documento.
% Programas :
% * cic       -- Graduação em Ciência da Computação
% * ecp       -- Graduação em Ciência da Computação
% * ppgc      -- Programa de Pós Graduação em Computação
% * pgmigro   -- Programa de Pós Graduação em Microeletrônica
%
% Tipos de Documento:
% * tc                -- Trabalhos de Conclusão (apenas cic e ecp)
% * diss ou mestrado  -- Dissertações de Mestrado (ppgc e pgmicro)
% * tese ou doutorado -- Teses de Doutorado (ppgc e pgmicro)
% * ti                -- Trabalho Individual (ppgc e pgmicro)
%
% Outras Opções:
% * english    -- para textos em inglês
% * openright  -- Força início de capítulos em páginas ímpares (padrão da
% biblioteca)
% * oneside    -- Desliga frente-e-verso
% * nominatalocal -- Lê os dados da nominata do arquivo nominatalocal.def

% Use unicode
\usepackage[utf8]{inputenc}   % pacote para acentuação

% Necessário para incluir figuras
\usepackage{graphicx}         % pacote para importar figuras

\usepackage{times}            % pacote para usar fonte Adobe Times
% \usepackage{palatino}
\usepackage{inconsolata}      % pacote para usar fonte Inconsolata em
                              % ambientes de código
% \usepackage{mathptmx}       % p/ usar fonte Adobe Times nas fórmulas

\usepackage{microtype}       % pacote para microtipografia

\usepackage[alf,abnt-emphasize=bf]{abntex2cite}	% pacote para usar citações abnt

% pacotes adicionais
\usepackage{amsmath}          % pacote para usar ambientes matemáticos
\usepackage{amssymb}
\usepackage{todonotes}

\usepackage{tikz}
\usetikzlibrary{arrows.meta, automata, positioning, quotes, fit}

\usepackage{xcolor}
\usepackage{placeins}
\usepackage{caption}
\usepackage{subcaption}
\usepackage{booktabs}
\usepackage{multirow}
\usepackage{multicol}
\usepackage{listings}
\usepackage{listings-rust}
\usepackage{svg}
\usepackage{mwe}
\usepackage{parcolumns}
\usepackage{mathtools}

%pacotes locais
\usepackage{bcprules}

%debug line, remove before submission
\overfullrule=5pt

%
% Informações gerais
%
\title{Porcelain: um Framework Semântico para Representar e Analisar Técnicas de Segurança de Memória.}
\translatedtitle{Porcelain: A Semantic Framework for Representing and Analyzing Memory Safety Techniques}

\author{Colle}{Pedro Henrique Boniatti}
% alguns documentos podem ter varios autores:
% \author{Flaumann}{Frida Gutenberg}
% \author{Flaumann}{Klaus Gutenberg}

% orientador e co-orientador são opcionais (não diga isso pra eles :))
\advisor[Prof.~Dr.]{Machado}{Rodrigo}
% \coadvisor[Prof.~Dr.]{Knuth}{Donald Ervin}

% a data deve ser a da defesa; se nao especificada, são gerados
% mes e ano correntes
% \date{maio}{2001}

% o local de realização do trabalho pode ser especificado (ex. para TCs)
% com o comando \location:
% \location{Itaquaquecetuba}{SP}

% itens individuais da nominata podem ser redefinidos com os comandos
% abaixo:
% \renewcommand{\nominataReit}{Prof\textsuperscript{a}.~Wrana Maria Panizzi}
% \renewcommand{\nominataReitname}{Reitora}
% \renewcommand{\nominataPRE}{Prof.~Jos{\'e} Carlos Ferraz Hennemann}
% \renewcommand{\nominataPREname}{Pr{\'o}-Reitor de Ensino}
% \renewcommand{\nominataPRAPG}{Prof\textsuperscript{a}.~Joc{\'e}lia Grazia}
% \renewcommand{\nominataPRAPGname}{Pr{\'o}-Reitora Adjunta de P{\'o}s-Gradua{\c{c}}{\~a}o}
% \renewcommand{\nominataDir}{Prof.~Philippe Olivier Alexandre Navaux}
% \renewcommand{\nominataDirname}{Diretor do Instituto de Inform{\'a}tica}
% \renewcommand{\nominataCoord}{Prof.~Carlos Alberto Heuser}
% \renewcommand{\nominataCoordname}{Coordenador do PPGC}
% \renewcommand{\nominataBibchefe}{Beatriz Regina Bastos Haro}
% \renewcommand{\nominataBibchefename}{Bibliotec{\'a}ria-chefe do Instituto de Inform{\'a}tica}
% \renewcommand{\nominataChefeINA}{Prof.~Jos{\'e} Valdeni de Lima}
% \renewcommand{\nominataChefeINAname}{Chefe do \deptINA}
% \renewcommand{\nominataChefeINT}{Prof.~Leila Ribeiro}
% \renewcommand{\nominataChefeINTname}{Chefe do \deptINT}

% A seguir são apresentados comandos específicos para alguns
% tipos de documentos.

% Relatório de Pesquisa [rp]:
% \rp{123}             % numero do rp
% \financ{CNPq, CAPES} % orgaos financiadores

% Trabalho Individual [ti]:
% \ti{123}     % numero do TI
% \ti[II]{456} % no caso de ser o segundo TI

% Monografias de Especialização [espec]:
% \espec{Redes e Sistemas Distribuídos}      % nome do curso
% \coord[Profa.~Dra.]{Weber}{Taisy da Silva} % coordenador do curso
% \dept{INA}                                 % departamento relacionado

%
% palavras-chave
% iniciar todas com letras maiúsculas, exceto no caso de abreviaturas
%
\keyword{Segurança de Memória}
\keyword{Semântica Formal}
\keyword{Rust}
\keyword{Borrow Checker}

% novos comandos
\newcommand{\db}[1]{\color{red} #1 \color{black}}
\newcommand{\gb}[1]{\color{purple} #1 \color{black}}

\newcommand{\OR}{\;|\;}
\newcommand{\NLOR}{|\;}
\newcommand{\St}[1]{\langle #1 \rangle}
\newcommand{\typearg}[1]{\text{<}#1\text{>}}
\newcommand{\KW}[1]{\texttt{#1}}
\newcommand{\FN}[1]{\texttt{#1}}
\newcommand{\CD}[1]{\texttt{#1}}

\newcommand{\ignoreLtex}[1]{#1}
\newcommand{\labelRule}[2]{
	\phantomsection #1 #2
}

% cores do lstlisting

\definecolor{codegreen}{rgb}{0,0.6,0}
\definecolor{codegray}{rgb}{0.5,0.5,0.5}
\definecolor{codepurple}{rgb}{0.58,0,0.82}
\definecolor{backcolour}{rgb}{0.95,0.95,0.92}

% NOTE: customizar a cor um pouco
\lstdefinestyle{mystyle}{
    backgroundcolor=\color{backcolour},   
    commentstyle=\color{codegreen},
    keywordstyle=\color{magenta},
    numberstyle=\tiny\color{codegray},
    stringstyle=\color{codepurple},
    basicstyle=\ttfamily\footnotesize,
    breakatwhitespace=false,         
    breaklines=true,                 
    captionpos=b,                    
    keepspaces=true,                 
    numbers=left,                    
    numbersep=5pt,                  
    showspaces=false,                
    showstringspaces=false,
    showtabs=false,                  
    tabsize=4
}
\renewcommand*{\lstlistingname}{Programa}

\lstset{style=mystyle, escapeinside={|.}{.}}

% NOTE: usado para o todonotes fazer o display correto
\setlength{\marginparwidth}{2.5cm}



%
% palavras-chave na lingua estrangeira
% iniciar todas com letras maiúsculas, exceto no caso de abreviaturas
%
\translatedkeyword{Memory Safety}
\translatedkeyword{Formal Semantics}
\translatedkeyword{Rust}
\translatedkeyword{Borrow Checker}

%\settowidth{\seclen}{1.10~}

%
% inicio do documento
%
\begin{document}

%\lstset{inputencoding=utf8/utf8}

\lstdefinelanguage{PCLback}{%
  sensitive%
, morecomment=[l]{//}%
, morecomment=[s]{/*}{*/}%
, moredelim=[s][{\itshape\color[rgb]{0,0,0.75}}]{\#[}{]}%
, morestring=[b]{"}%
, alsodigit={}%
, alsoother={}%
, alsoletter={!}%
%
%
% [1] reserve keywords
% [2] mem keywords
% [3] pcodes 
, morekeywords={if, else, while, let, global}  
, morekeywords=[2]{malloc, free, panic}  % mem funcs
, morekeywords=[3]{OutOfBoundsRead, OutOfBoundsWrite, NullPtrDereference,UserError, UseAfterFree, UninitializedAcess, FreeMemoryNotOnHeap, PartialFree,DoubleFree
}
}

\lstdefinelanguage{PCLfront}{%
  sensitive%
, morecomment=[l]{//}%
, morecomment=[s]{/*}{*/}%
, moredelim=[s][{\itshape\color[rgb]{0,0,0.75}}]{\#[}{]}%
, morestring=[b]{"}%
, alsodigit={}%
, alsoother={}%
, alsoletter={!}%
%
%
% [1] reserve keywords
% [2] mem keywords
, morekeywords={if, else, let, var, alias}  
, morekeywords=[2]{new, delete, stop, nullalias, nullptr}  % mem funcs
}
% folha de rosto
% às vezes é necessário redefinir algum comando logo antes de produzir
% a folha de rosto:
% \renewcommand{\coordname}{Coordenadora do Curso}
\maketitle

% dedicatoria
 \clearpage
 \begin{flushright}
     \mbox{}\vfill
     {\sffamily\itshape
       ``
       Eu corro o risco de ficar como as pessoas grandes, 
	   que só se interessam por números.''\\
	   %ou Gosto que levem a sério as minhas desgraças.
     }
     --- \textsc{Antoine de Saint-Exupéry}
        % \vfill
 \end{flushright}

% agradecimentos
\chapter*{Agradecimentos}
A Célio e Marli, nada mais do que todos os agradecimentos 
do mundo por serem meus pais e por me apoiarem nessa jornada. Muito obrigado ao Vinicius, João Pedro e Eduardo, por revisarem o meu TCC e estarem comigo nos últimos dias de sufoco antes da entrega. Obrigado a Sofia por todo o apoio e amor nesses anos. Obrigado ao Tales, Tomas e Pedro, por serem família mesmo não sendo família. Por fim, dedico esse texto ao Snooby, que não pôde ler isso; e ao Chico, que não pode ler isso.


% resumo na língua do documento
\begin{abstract}
	Muitos \emph{designers} de linguagens de programação modernas notam que uma estratégia de segurança de memória é necessária para a linguagem ser levada em consideração. Mas como validar essas estratégias? Como prová-las corretas? Esse trabalho estabelece Porcelain: um \emph{framework} semântico para representar e analisar técnicas de segurança de memória. Nele, elabora-se sobre as falhas de memória, propondo uma linguagem de programação que as exemplifica. Para demonstrar a utilidade desse \emph{framework}, desenvolve-se uma das estratégias de memória com o intuito de demonstrar as capacidades de prova do modelo.
\end{abstract}

% resumo na outra língua
\begin{translatedabstract}
Many modern programming language designers have noticed that a memory safety strategy is necessary for a language to be taken into consideration. But how to validate those strategies? How to prove them correct? This project aims to establish Porcelain: a semantic framework for representing and analyzing memory safety techniques. In it, memory faults are elaborated about, for the purpose of proposing a programing language that exemplifies them. To demonstrate the utility of this framework, it was developed one memory safety strategy in order to demonstrate the model's proof capabilities.
\end{translatedabstract}

% lista de figuras
\listoffigures

% lista de tabelas
\listoftables

% lista de abreviaturas e siglas
% o parametro deve ser a abreviatura mais longa
% A NBR 14724:2011 estipula que a ordem das abreviações
% na lista deve ser alfabética (como no exemplo abaixo).
\begin{listofabbrv}{ASCII}
    \item[API] \emph{Application Programming Interface} (Interface Programável de Aplicação)
	\item[ASCII] \emph{American Standard Code for Information Interchange} (Código Padrão Americano para o Intercâmbio de Informação)
    \item[AST] \emph{Abstract Syntax Tree} (Árvore de Sintaxe Abstrata)
    \item[CFG] \emph{Control Flow Graph} (Grafo de Controle de Fluxo)
    \item[GC] \emph{Gabage Collection} (Coletor de Lixo)
    \item[OS] \emph{Operacional System} (Sistema Operacional)
    \item[PCL] \emph{Porcelain}
    \item[UB] \emph{Undefined Behavior} (Comportamento Indefinido)
\end{listofabbrv}

% segundo argumento do begin é o termo de maior tamanho do item, para o espaçamento
% \begin{listofsymbols}{$\alpha\beta\pi\omega$}
%     \item[$\Gamma$] Ambiente de nomes de variáveis a tipos 
%     \item[$U$] Ambiente de variáveis usadas
%     \item[$R$] Fechamento binário entre regiões.
%     \item[$L$] Fechamento binário entre regiões e conjunto de concessões.
% \end{listofsymbols}

% sumario
\tableofcontents

% aqui comeca o texto propriamente dito

% introducao
% 1. Introdução
\chapter{Introdução}

% The C programming language\cite{CLANG} is the foundation of systems programming. 
% Up to the adoption of Rust in early 2024\cite{RUSTFORLINUX}, 
% it was the (only) language of the Linux kernel. 
% Despite this great importance, C is still prone to many types of memory safety violations, 
% with very little guarantees from the compiler.

% At the start of the millennium, Cyclone\cite{CYCLONE1} started the trend of low-level
% programming languages with safe memory systems without the broad use of 
% Garbage Collection (GC). 
% This was followed by CCured\cite{CCURED}, Rust\cite{RUSTBOOK} and more, 
% each with its own memory safety solution. 

%	1.1 Motivação
\section{Motivação}

% Given a memory safety solution, it is important for it to be formally proven correct, 
% as was done by \citet{RUSTBELT} and \citet{RUSTSYMBOLIC} for Rust's borrow checker.
% With that, this work introduces Porcelain: a Semantic Framework for Representing 
% and Analyzing Memory Safety Techniques. 
% The aim is to define a front-end language by compilation to porcelain's defined 
% back-end language as a means to prove certain theorems about the front-end languages 
% memory safety mechanisms.


%	1.2 Related Work
\section{Trabalhos Relacionado}

%	1.3 Objetivos
\section{Objetivos}

%	1.4 Organização
\section{Organização}

%This project in its current state includes a back-end language defined
% by a small-step operational semantics as well as a front-end language,
% modeling a borrow checker, defined via compilation to the back-end language.

%\todo[inline]{Não tem necessidade de fazer essa sessão agora, muito menos o capítulo.}
% 1. Introdução
%	1.1 Motivação
%	1.2 Related Work
%	1.3 Objetivos
%	1.4 Organização
% 2. Background
\chapter{Background}

%	2.1 O que é Semantica Operacional
\section{Semântica Operacional}

%   2.2 Erros de memória
\section{Erros de Memória}

% Memory safety bugs, as it concerns low level languages, can be subdivided in 5 classes: 
% Spacial, Temporal, Type, Initialization and Data-Race Safety\cite{Apple22,Google24}. 
% Even though memory safety bugs can be subdivided in many more ways \cite{7KINGDOMS,CWELIST},
% each of those classes maps to a distinct mechanism for solving them. 
% Spacial safety can be solved with dependent types \cite{tarditi2018checked} and 
% runtime bounds checks\cite{CYCLONE1}; Temporal safety can be solved with 
% Garbage Collection (GC), memory regions\cite{REGMEM}, a static alias 
% analyzer\cite{Stjerna1684081}, key-lock systems \cite{FATPOINTERS, CCURED}, 
% and more. Both Type and Initialization can be solved by imposing stricter
% constraints on declarations and type conversions. Data-Race safety is beyond
% the scope of this project.


\todo[inline]{Adicionar aqui alguma coisa que faltar, mas a parte dos tipos lineares e borrow checker ficam pra depois no texto.}
% 2. Background
%	2.1 O que é Semantica Operacional
%   2.2 Erros de memória
% 3. PCL back
\chapter{PCL-back}



%	3.1 Syntax
\section{Sintaxe}
\begingroup
\setlength{\jot}{-0.2ex} 
	\begin{align*}
		Locals \ni l ::&= l^m \OR l^p &&\\ 
		Value \ni v ::&= n \OR l && \\
		BinOp \ni op ::&= + | - | * | < | > | = | \land | \lor \\
		Expression \ni e ::&= x \OR v \\
		&\NLOR e\; op\; e \OR !e  \\
		&\NLOR \text{*}e \OR \&x \\
		&\NLOR e;e \; \OR \{\,e\,\} \; \\ 
		&\NLOR \mathbf{malloc}(e) \OR \mathbf{free}(e, e) \; \\ 
		&\NLOR \mathbf{let}\; x[n] \OR e := e \; \\
		&\NLOR \mathbf{if}(e) \; e \; \mathbf{else} \; e \; \OR \mathbf{while}(e) \; e \\
		&\NLOR f(\bar e) \\ 
		&\NLOR \mathbf{panic}\;pcode \\ 
		PanicCodes \ni pcodes ::&= \mathbf{OutOfBoundsRead}\\
		&\NLOR \mathbf{OutOfBoundsWrite}\\
		&\NLOR \mathbf{NullPtrDereference}\\
		&\NLOR \mathbf{UserError}\\
		&\NLOR \mathbf{UseAfterFree}\\
		&\NLOR \mathbf{UninitializedAcess}\\
		&\NLOR \mathbf{FreeMemoryNotOnHeap}\\
		&\NLOR \mathbf{PartialFree}\\
		&\NLOR \mathbf{DoubleFree}\\
		Function \ni F ::&= \mathbf{let} \; f(\overline{x})\; e \; F \; | \; \mathbf{let}\;() \; e \\
		Globals \ni G ::&= \mathbf{global}\; x[n] \;G \;|\; F
	\end{align*}
\endgroup


%	3.2 Memory Model
\section{Modelo de Memória}

\todo[inline]{Fazer essa uma sessão própria ou juntar com o texto anterior?}
\subsection{Ponteiros}

\subsection{Pilha}

\subsection{Heap}

%	3.3 Operational Semantics
\section{Semântica Operacional}
 \todo[inline]{Dar uma formatada mais legal nessas regras que tá meio feio}

\infrule[Compose-s]
    {F \vdash \St{e_1,a, p, m} \to \St{e_1',a', p', m'}}
    {F \vdash \St{e_1;e_2,a, p, m} \to \St{e_1';e_2,a', p', m'}}

\infrule[Compose]
    {}
    {F \vdash \St{v;e, a, p, m} \to \St{e, a, p, m}}
\infrule[Escopo]
    {}
    {F \vdash \St{\{e\},(a,g), p, m} \to \\ \St{\mathbf{pop}\; e,(\{\} : a, g),stack : p, m}}
\infrule[Pop-s]
    {F \vdash \St{e,a, p, m} \to \St{e',a', p', m'}}
    {F \vdash \St{\mathbf{pop}\;e,a, p, m} \to \St{\mathbf{pop}\;e',a', p', m'}} 
 \infrule[Pop]
    {\mathbf{pop}_{stack}(p) = p'}
    {F \vdash \St{\mathbf{pop}\;v,(frame : a, g), p, m} \to\\\St{v,(a, g), p', m}}

\infrule[Let]
    {\mathbf{top}(p) = i \andalso \mathbf{newkey}() = k \andalso l = l^p\{i, k, 0, n\}}
    {F \vdash \St{\mathbf{let}\; x[n],(fr : a, g), p, m} \to\\ \St{l, (fr[x \mapsto l] : a, g),(\bot, k)_n : p, m}} 

\infrule[Atribui-deref-ls]
    {F \vdash \St{e_1,a, p, m} \to \St{e_1', a', p', m'}}
    {F \vdash \St{\text{*}e_1 := e_2, a, p, m} \to \St{\text{*}e_1' := e_2, a', p', m'}} 

\infrule[Atribui-var-rs]
    {F \vdash \St{e, a, p, m} \to \St{e', a', p', m'}}
    {F \vdash \St{x := e, a, p, m} \to \St{x := e', a', p', m'}} 

\infrule[Atribui-deref-rs]
    {F \vdash \St{e,a, p, m} \to \St{e', a', p', m'}}
    {F \vdash \St{\text{*}l := e, a, p, m} \to \St{\text{*}l := e', a', p', m'}} 

\infrule[Atribui-var-obe]
    {a(x) = l^p\{i, k, o, s\} \andalso 0 > o \ge s}
    {F \vdash \St{x := v, a, p, m} \to \\ \St{\mathbf{panic}\;\mathbf{OutOfBoundsWrite}, a, p, m}}
    
\infrule[Atribui-var-uaf]
    {a(x) = l^p\{i, k, o, s\} \\ 0 \leq o < s\andalso p(i + o) = (v_p, k_p) \andalso k \neq k_p }
    {F \vdash \St{x := v, a, p, m} \to \\ \St{\mathbf{panic}\;\mathbf{UseAfterFree}, a, p, m}}

\infrule[Atribui-var]
    {a(x) = l^p\{i, k, o, s\} \\ 0 \leq o < s\quad p(i + o) = (v_p, k_p) \quad k = k_p }
    {F \vdash \St{x := v, a, p, m} \to \St{v, a, p[i + o \mapsto v], m}} 

\infrule[Atribui-deref-pilha-obe]
    {0 > o \ge s}
    {F\vdash\St{\text{*}l^p\{i, k, o, s\} := v, a, p, m} \to \\ \St{\mathbf{panic}\;\mathbf{OutOfBoundsWrite}, a, p, m}} 

\infrule[Atribui-deref-pilha-uaf]
    {0 \leq o < s\quad p(i + o) = (v_p, k_p) \quad k \neq k_p }
    {F\vdash\St{\text{*}l^p\{i, k, o, s\} := v, a, p, m} \to\\ \St{\mathbf{panic}\;\mathbf{UseAfterFree}, a, p, m}} 
    
\infrule[Atribui-deref-pilha]
    {0 \leq o < s\quad p(i + o) = (v_p, k_p) \quad k = k_p }
    {F\vdash\St{\text{*}l^p\{i, k, o, s\} := v, a, p, m} \to\\ \St{v, a, p[i + o \mapsto v], m}} 

\infrule[Atribui-deref-mem-npd]
    {i = 0}
    {F\vdash\St{\text{*}l^m\{i, k, o, s\} := v, a, p, m} \to\\ \St{\mathbf{panic}\;\mathbf{NullPtrDeref}, a, p, m}} 

\infrule[Atribui-deref-mem-obe]
    {i \neq 0 \quad 0 > o \ge s}
    {F\vdash\St{\text{*}l^m\{i, k, o, s\} := v, a, p, m} \to\\ \St{\mathbf{panic}\;\mathbf{OutOfBoundsWrite}, a, p, m}} 

\infrule[Atribui-deref-mem-uaf]
    {i \neq 0 \quad 0 \leq o < s\quad m(i + o) = (v_m, k_m) \quad k \neq k_m }
    {F\vdash\St{\text{*}l^m\{i, k, o, s\} := v, a, p, m} \to\\ \St{\mathbf{panic}\;\mathbf{UseAfterFree}, a, p, m}} 
    
\infrule[Atribui-deref-mem]
    {i \neq 0 \quad 0 \leq o < s\quad m(i + o) = (v_m, k_m) \quad k = k_m }
    {F\vdash\St{\text{*}l^m\{i, k, o, s\} := v, a, p, m} \to \\\St{v, a, p, m[i + o \mapsto v]}}

\infrule[Malloc-s]
    {F\vdash\St{e, a, p, m} \to\St{e', a', p', m'}}
    {F\vdash\St{\mathbf{malloc}(e),a, p, m} \to\St{\mathbf{malloc}(e'),a', p', m'}}
    
\infrule[Malloc]
    {\mathbf{loc}(m, n) = i \quad \mathbf{newkey}() = k \quad l = l^m\{i, k, 0, n\}}
    {F\vdash\St{\mathbf{malloc}(n),a, p, m} \to\St{l, a, p, m[i_{0..n} \mapsto (\bot,k)]}} 
    
\infrule[Free-ls]
    {F\vdash\St{e_1, a, p, m} \to\St{e_1', a', p', m'}}
    {F\vdash\St{\mathbf{free}(e_1, e_2),a, p, m} \to\St{\mathbf{free}(e_1', e_2),a', p', m'}}
    
\infrule[Free-rs]
    {F\vdash\St{e, a, p, m} \to\St{e', a', p', m'}}
    {F\vdash\St{\mathbf{free}(l, e),a, p, m} \to\St{\mathbf{free}(l, e'),a', p', m'}} 

\infrule[Free-mnh]
    {}
    {F\vdash\St{\mathbf{free}(l^p, n),a, p, m} \to\\ \St{\mathbf{panic}\;\mathbf{FreeMemoryNotOnHeap},a, p, m}}

\infrule[Free-memory-null] 
    {i = 0}
    {F\vdash\St{\mathbf{free}(l^m\{i, k, o, s\}, n),a, p, m} \to\\ \St{\mathbf{panic}\;\mathbf{FreeMemoryNotOnHeap},a, p, m}}
    
\infrule[Free-pf] 
    {i \neq 0 \quad o \neq 0 \lor s \neq n}
    {F\vdash\St{\mathbf{free}(l^m\{i, k, o, s\}, n),a, p, m} \to\\ \St{\mathbf{panic}\;\mathbf{PartialFree},a, p, m}}
    
\infrule[Free-df]
    {i \neq 0 \quad o = 0 \quad s = n \quad m(i) = (v_m, k_m) \quad k \neq k_m}
    {F\vdash\St{\mathbf{free}(l^m\{i, k, o, s\}, n),a, p, m} \to\\ \St{\mathbf{panic}\;\mathbf{DoubleFree},a, p, m}}

\infrule[Free]
    {i \neq 0 \quad o = 0 \quad s = n \quad m(i) = (v_m, k_m) \quad k = k_m}
    {F\vdash\St{\mathbf{free}(l^m\{i, k, o, s\}, n),a, p, m} \to\St{n,a, p, m[i_{0..n} \mapsto (-,0)]}}

\infrule[BinOp-ls]
    {F \vdash \St{e_1,a, p, m} \to \St{e_1', a', p', m'}}
    {F \vdash \St{e_1 \;op\; e_2, a, p, m} \to \St{e_1' \;op\; e_2, a', p', m'}} 

\infrule[BinOp-rs]
    {F \vdash \St{e,a, p, m} \to \St{e', a', p', m'}}
    {F \vdash \St{v \;op\; e, a, p, m} \to \St{v \;op\; e', a', p', m'}}

\infrule[BinOp]
    {\mathbf{binop}(op, v_1, v_2) = v'}
    {F \vdash \St{v_1 \;op\; v_2, a, p, m} \to \St{v', a, p, m}}

\infrule[Not-s]
    {F \vdash \St{e,a, p, m} \to \St{e', a', p', m'}}
    {F \vdash \St{!e, a, p, m} \to \St{!e', a', p', m'}}

\infrule[Not]
    {\mathbf{not}(v) = n}
    {F \vdash \St{!v, a, p, m} \to \St{n, a, p, m}}

\infrule[Var-obr]
    {a(x) = l^p\{i, k, o, s\} \quad 0 > o \ge s}
    {F \vdash \St{x, a, p, m} \to\\ \St{\mathbf{panic}\;\mathbf{OutOfBoundsRead}, a, p, m}}
    
\infrule[Var-uaf]
    {a(x) = l^p\{i, k, o, s\} \\ 0 \leq o < s\quad p(i + o) = (v_p, k_p) \quad k \neq k_p}
    {F \vdash \St{x, a, p, m} \to \St{\mathbf{panic}\;\mathbf{UseAfterFree}, a, p, m}}

\infrule[Var-ua]
    {a(x) = l^p\{i, k, o, s\} \\ 0 \leq o < s\quad p(i + o) = (v_p, k_p) \quad k = k_p \quad v_p = \bot}
    {F \vdash \St{x, a, p, m} \to \\ \St{\mathbf{panic}\;\mathbf{UninitializedAcess}, a, p, m}} 

\infrule[Var]
    {a(x) = l^p\{i, k, o, s\} \\ 0 \leq o < s\quad p(i + o) = (v_p, k_p) \quad k = k_p \quad v_p \neq \bot}
    {F \vdash \St{x, a, p, m} \to \St{v_p, a, p, m}}

\infrule[Ref]
    {a(x) = l^p}
    {F \vdash \St{\&x, a, p, m} \to \St{l^p, a, p, m}}
    
\infrule[Deref-s]
	{F \vdash \St{e,a, p, m} \to \St{e', a', p', m'}}
	{F \vdash \St{\text{*}e, a, p, m} \to \St{\text{*}e', a', p', m'}} 

\infrule[Deref-pilha-obr]
    {0 > o \ge s}
    {F \vdash \St{\text{*}l^p\{i, k, o, s\}, a, p, m} \to \\ \St{\mathbf{panic}\;\mathbf{OutOfBoundsRead}, a, p, m}}
    
\infrule[Deref-pilha-uaf]
    {0 \leq o < s\quad p(i + o) = (v_p, k_p) \quad k \neq k_p}
    {F \vdash \St{\text{*}l^p\{i, k, o, s\}, a, p, m} \to\\ \St{\mathbf{panic}\;\mathbf{UseAfterFree}, a, p, m}} 

\infrule[Deref-pilha-ua]
    {0 \leq o < s\quad p(i + o) = (v_p, k_p) \quad k = k_p \quad v_p = \bot}
    {F \vdash \St{\text{*}l^p\{i, k, o, s\}, a, p, m} \to\\ \St{\mathbf{panic}\;\mathbf{UninitializedAcess}, a, p, m}}

\infrule[Deref-pilha]
    {0 \leq o < s\quad p(i + o) = (v_p, k_p) \quad k = k_p \quad v_p \neq \bot}
    {F \vdash \St{\text{*}l^p\{i, k, o, s\}, a, p, m} \to \St{v_p, a, p, m}}

\infrule[Deref-memória-null]
    {i = 0}
    {F \vdash \St{\text{*}l^m\{i, k, o, s\}, a, p, m} \to\\ \St{\mathbf{panic}\;\mathbf{NullPtrDeref}, a, p, m}}

\infrule[Deref-memória-obr]
    {i \neq 0 \quad 0 > o \ge s}
    {F \vdash \St{\text{*}l^m\{i, k, o, s\}, a, p, m} \to\\ \St{\mathbf{panic}\;\mathbf{OutOfBoundsRead}, a, p, m}}
    
\infrule[Deref-memória-uaf]
    {i \neq 0 \quad 0 \leq o < s\quad m(i + o) = (v_m, k_m) \quad k \neq k_m}
    {F \vdash \St{\text{*}l^m\{i, k, o, s\}, a, p, m} \to\\ \St{\mathbf{panic}\;\mathbf{UseAfterFree}, a, p, m}}

\infrule[Deref-memória-ua]
    {i \neq 0 \quad 0 \leq o < s\quad m(i + o) = (v_m, k_m) \quad k = k_m \quad v_m = \bot}
    {F \vdash \St{\text{*}l^m\{i, k, o, s\}, a, p, m} \to\\ \St{\mathbf{panic}\;\mathbf{UninitializedAcess}, a, p, m}} 
    
\infrule[Deref-memória]
    {i \neq 0 \quad 0 \leq o < s\quad m(i + o) = (v_m, k_m) \quad k = k_m \quad v_m \neq \bot}
    {F \vdash \St{\text{*}l^m\{i, k, o, s\}, a, p, m} \to \St{v_m, a, p, m}}

\infrule[If-s]
    {F \vdash \St{e_1,a, p, m} \to \St{e_1', a', p', m'}}
    {F \vdash \St{\mathbf{if}\;(e_1)\; e_2\; \mathbf{else}\; e_3 ,a, p, m} \to\\ \St{\mathbf{if}\;(e_1')\; e_2\; \mathbf{else}\; e_3 ,a', p', m'}}
    
\infrule[If-true]
    {\mathcal{N}(v) \neq 0}
    {F \vdash \St{\mathbf{if}\;(v)\; e_1\; \mathbf{else}\; e_2 ,a, p, m} \to \St{e_1, a, p, m}} 
\infrule[If-false]
    {\mathcal{N}(v) = 0}
    {F \vdash \St{\mathbf{if}\;(v)\; e_1\; \mathbf{else}\; e_2 ,a, p, m} \to \St{e_2, a, p, m}} 
    
\infrule[While]
    {}
    {F \vdash \St{\mathbf{while}\;(e_1)\; e_2, a, p, m} \to \\ \St{\mathbf{if}\;(e_1)\; (e_2;\mathbf{while}\;(e_1)\; e_2)\; \mathbf{else}\; 0, a, p, m}} 

\infrule[LetFunction-collect]
    {}
    {\St{\mathbf{let} \; f(\overline{x_i})\; e \; F_0 ,F} \to_F \St{F_0, F[f \mapsto \St{[\overline{x_i}], e}]}}

\infrule[Globals-collect]
    {\mathbf{newkey}() = k \quad \mathbf{top}(p) = i}
    {\St{\mathbf{global} \; x[n]\; G , g, p} \to_G\\ \St{G, g[x \mapsto l^p \{i, k, 0, n\}],(\bot, k)_n : p,}} 

\infrule
    {\St{P, \{\}, []} \to_G^* \St{F_0, g, p}\quad\St{F_0, \{\}} \to_F^* \St{\mathbf{let}\;()\;e , F}}
    {F \vdash \St{\{e\}, ([],g),p,[]} \to^* \St{v, ([],g), p, m}} 

%	3.4 Error detection
\section{Detecção de Erros}
% 3. PCL back
%	3.1 Syntax
%	3.2 Memory Model
%	3.3 Operational Semantics
%	3.4 Error detection
% 4. Use Case: Borrow Checker (explica o borrow checker e o algoritmo do polonius)
\chapter{Caso de Uso: Borrow Checker}
% O que é e o porque de um borrow checker
% One of the most researched solutions for memory bugs has been 
% Rust's borrow checker\cite{RUSTBOOK}. It promises a zero overhead Temporal Safety 
% solution by constraining the ability to alias and adding code annotations 
% to validate temporal memory access. Although, properly validating its claims 
% has been a challenge\cite{RUSTBELT, RUSTSYMBOLIC}.

% In the case of $PCL_{back}$, there is an interesting bijection between the key locks 
% used to check for temporal safety and the lifetimes used for computing the correctness 
% of an access. Therefore, one could postulate that to implement a front-end language 
% with a borrow checking system $PCL_{front}$, it could be proven that 
% no \textit{Use After Free} errors would occur in $PCL_{back}$, 
% given a correct input program and a sound compilation.

%	4.1 Rust and the Borrow Checker
\section{O Borrow Checker do Rust}

%	4.2 Tipos Lineares
\section{Tipos Lineares}
% Call back da sessão anterior, recontextualizando o comportamento do borrow checker 
% do Rust como tipos lineares/Afim

%	4.3 Polonius
\section{Polonius}
% Explicar os detalhes da implementação do polonius
% Expecífico ao domínio do Rust
% 4. Use Case: Borrow Checker (explica o borrow checker e o algoritmo do polonius)
%	4.1 Rust and the Borrow Checker
%	4.2 Tipos Lineares
%	4.3 Polonius
% 5. PCL front
\chapter{PCL-front}

% 	5.1 PCL front's Borrow Checker
\section{O Borrow Checker de PCL-front}

% 	5.2 Syntax
\section{Sintaxe}

\begingroup
\setlength{\jot}{-0.2ex} 
	\begin{align*}
		Types \ni \tau ::&= \KW{int} \OR \text{*}\tau \OR @\tau'a \OR (\bar\tau) \to \tau \\
		Locals \ni l ::&= l^m \OR l^p &&\\ 
		Value \ni v ::&= n \OR l && \\
		BinOp \ni op ::&= + | - | * | < | > | = | \land | \lor \\
		Expression \ni e ::&= x \OR v \\
		&\NLOR e\; op\; e \OR !e  \\
		&\NLOR \text{*}e \OR \&x \OR \KW{alias} \; x \OR \KW{alias}\text{*} \; x\\
		&\NLOR e;e \OR \{e\} \\ 
		&\NLOR \KW{new}\typearg\tau(e) \OR \KW{delete}(e, e) \; \\ 
		&\NLOR \KW{var}\; x: \tau := e \OR e_1 := e_2 \\
		&\NLOR e_1 :=: e_2 \OR e_1 :=:\!\text{*}\;e_2\\
		&\NLOR \KW{if}(e) \; e \; \KW{else} \; e \; \OR \KW{while}(e) \; e \\
		&\NLOR f(\bar e) \\ 
		&\NLOR \KW{stop}\\ 
		&\NLOR \KW{nullprt}\typearg{\tau} \OR \KW{nullalias}\typearg{\tau}\\ 
		Function \ni F ::&= \KW{fn} \; f(\overline{x : \tau}) \to \tau \; e \; F \; | \; \mathbf{let}\;() \; e \\
	\end{align*}
\endgroup

% 	5.3 Checagem de Tipos
\section{Checagem de Tipos e Regiões}

%   5.4 Definição via compilação
\section{Definição via Compilação}
% 5. PCL front
% 	5.1 PCL front's Borrow Checker
% 	5.2 Syntax
% 	5.3 Checagem de Tipos
%   5.4 Definição via compilação
%\include{chapters/6-Propriedades.tex}
% 6. Propriedades
% 7. Conclusão
\chapter{Conclusão}
\label{chap7}

Este projeto se iniciou visando definir um sistema para provar métodos de segurança de memória. Esse desenvolvimento estendeu-se até a definição da linguagem de \emph{front-end}, $PCL_{front}$. Com ela, pode-se definir uma semântica que evita falhas de \emph{Use-After-Free} utilizando um sistema de \emph{Borrow Checker} com o algoritmo Polonius. Esse sistema se inspirou também em outros métodos para adequá-lo melhor ao contexto de implementação.

A construção de $PCL_{back}$ é expressiva suficiente para modelar todas as falhas de memória com claridade. Durante este texto, foram usados alguns exemplos dessa expressividade (\ref{lst:pclback:spacial}), levando em conta a ocorrência dessas falhas em C. Mais exemplos desses paralelos foram construídos entre os erros de C e $PCL_{back}$, mas foram omitidos por espaço. Uma série deles existe na publicação Erros de Memória e Porcelain no blog do autor\footnote{\url{https://sacolle.github.io/blog/posts/porcelain-emulando-os-erros-de-memoria-de-c/}}, que foi desenvolvido paralelamente a este trabalho.

Este projeto serve como um estudo sobre falhas de memória, suas definições, problemas e soluções. Ao construir as linguagens $PCL_{back}$ e $PCL_{front}$ pode-se se aproximar mais desses conceitos, aplicando eles de forma prática e direta. Nesse processo, almejou-se construir novas técnicas e reformular o que é uma falha de memória. A relevância deste projeto é essa visão mais micro dessas falhas, as suas minúcias e desafios quando as tenta solucionar.

% falar aqui da do sistema em haskell
\section{Resultados Obtidos}

A AST (árvore sintática abstrata) das linguagens, e as suas respectivas avaliações, foram implementadas em Haskell. A sintaxe da linguagem se provou muito próxima da anotação matemática, permitindo rápida iteração entre mudança de anotação e mudança de código e vice e versa. Para $PCL_{back}$ foi também desenvolvido um \emph{lexer} e um \emph{parser}, também em Haskell, que auxiliaram muito na prototipação da linguagem e na formulação de exemplos. A adição de declarações globais surgiram desse processo de iteração, em que se observou que seria necessário para demonstrar as falhas associadas a estruturas globais de alocação de memória. A formatação das linguagens também as posiciona bem caso o projeto progredir para o uso de um sistema de provas formais, como Coq \cite{COQ} ou Lean \cite{LEAN4}, que usam uma notação funcional similar a Haskell.
% 	7.1 Limitações
\section{Limitações}

Devido às limitações de escopo deste projeto, não foi possível atingir todos os almejos deste projeto. Mesmo com $PCL_{front}$ tendo sua funcionalidade reduzida para caber no escopo, não foi possível chegar no estágio de validação dessa linguagem. Precisa-se, ainda, realizar testes e validar o modelo de \emph{alias} da linguagem a fim de validar que não existem erros lógicos na formação deste. Por consequência, não se conseguiu progredir para a etapa de prova do trabalho, em que se tentaria provar formalmente a segurança de memória de $PCL_{front}$ quanto erros de \emph{Use-After-Free}.

A linguagem $PCL_{back}$, por sua vez, tem sua expressividade limitada devido a certas decisões que facilitaram a sua implementação. A maior delas foi como foram implementadas alocações com $\KW{malloc}$ e liberações com o $\KW{free}$. O fato de $\KW{free}$ necessitar da quantidade de espaços a serem liberados gera padrões que não possuem correspondentes em C. Isso fez com que as definições de erros temporais que corrompem a função \lstinline[language=C]|free| em C sejam mais arbitrárias que o desejado na sua reprodução em $PCL_{back}$. Além disso, a linguagem não possui nenhuma proteção contra acessos fora dos limites de estruturas dentro de \emph{buffers}. Caso se esteja operando sobre estruturas em listas, pode-se não intencionalmente sobrescrever membros de uma estrutura da lista ao acessar outro. Outra limitação que estende dessa é que $PCL_{back}$ não é tipado. Esse fato simplifica a compilação a ele, mas omite certas informações para captura de certas falhas mais veladas. Além disso, $PCL_{back}$ depende da semântica da linguagem para definir a compatibilidade entre tipos, que pode ser prejudicial na hora de realizar as provas estruturais.

% 	7.2 Future Work
\section{Trabalhos Futuros}

O futuro deste trabalho está bem traçado, complementar e validar $PCL_{back}$, possivelmente reformulando alguns conceitos, como a ampla mutabilidade dos \emph{alias}, e depois provar que o sistema previne falhas de \emph{Use-After-Free}. É de interesse também gerar uma prova que a compilação para $PCL_{back}$ é correta. Não é fora do escopo talvez gerar uma linguagem de \emph{front-end} que implemente uma solução de memória mais simples para testar o processo e fazer modificações necessárias nele, caso a prova do sistema de $PCL_{front}$ se demonstre muito penosa.

Uma aspiração grandiosa deste trabalho é fazer várias linguagens de \emph{front-end}, refinando o trabalho todo no processo de prová-las através da compilação para $PCL_{back}$. Qualquer progresso no trabalho será publicado no blog do Autor\footnote{\url{https://sacolle.github.io/blog/}} , que com sorte será periodicamente. 
% 7. Conclusão
% 	7.1 Limitações
% 	7.2 Future Work


% referências
% aqui será usado o environment padrao `thebibliography'; porém, sugere-se
% seriamente o uso de BibTeX e do estilo abnt.bst (veja na página do
% UTUG)
%
% observe também o estilo meio estranho de alguns labels; isso é
% devido ao uso do pacote `natbib', que permite fazer citações de
% autores, ano, e diversas combinações desses

\bibliographystyle{abntex2-alf}
\bibliography{references}

\end{document}
